\chapter{Ziele}
Ins Billard-Spiel einzusteigen ist nicht ganz einfach. Zu Beginn lässt es sich schlecht abschätzen,
welchen Weg eine Kugel nehmen wird, wenn man sie anschlägt und den optimalen Stoss über mehrere Züge hinaus zu planen,
erst recht. Denn bei fortgeschrittenen Spielen ist es oft wichtig, dass die weisse Kugel optimal für den nächsten Stoss
platziert wird.

Es soll ein System entstehen, das dem Spieler den optimalen Stoss vorschlägt basierend auf Kriterien und
optional einer festgelegten Tiefe des Spielstands. Dazu soll eine Kamera den Spielstand auf dem Billiartisch erkennen, verarbeiten und
dem Spieler Hilfestellungen mittels eines Projektors anzeigen.

Diese Arbeit setzt auf bereits geleisteter Tätigkeit aus \glqq Projekt 2\grqq{} auf\cite{project2:ziele}. Die
Tätigkeiten werden nachfolgend beschrieben.
\begin{description}
    \item[Vorschlag eines optimalen Stosses] Es wird ein Stoss vorgeschlagen, welcher anhand der gewählten Kugel
    möglichst optimal ist. Dieser Stoss kann direkt (einfach) oder indirekt (erweitert) sein.
\end{description}

Es ist weiterhin anzumerken, dass es in erster Linie um Snooker-Billard geht. Dies hat mehrere Gründe. Einerseits soll
in dieser Arbeit nicht die Klassifikation der Kugeln im Zentrum stehen, sondern die Suche nach einem optimalen Stoss.
Es wird angenommen, dass dies mit Snooker-Kugeln einfacher ist als mit Pool-Billard-Kugeln. Andererseits wird das
Projekt zusammen mit einem Unternehmen durchgeführt, welches eventuell auch einen kommerziellen Ansatz verfolgen
will. Da grössere Turniere wie Weltmeisterschaften in Snooker ausgetragen werden, kam schnell der Wunsch auf, das
Hauptaugenmerk darauf zu legen. Nichtsdestotrotz wird die Anwendung so abstrakt gehalten, dass sie mit wenig Aufwand
auf Pool-Billard portiert werden könnte. Dies bildet jedoch kein Ziel der Bachelor-Thesis.

\newpage

\subsection{Planung}
Die initiale Planung beinhaltet eine Auflistung der Tätigkeiten, dem zugewiesenen Meilenstein sowie
deren Schätzung in PT (Personen-Tage). Jedem Arbeitspaket wird eine ID zugewiesen,
welche bei der Zeiterfassung verlinkt wird. Das Total der zu vergebenden PT beträgt 90.

\begin{table}[ht]
    \rowcolors{1}{\seccolor!10}{\seccolor!10} % Rows with 10% of secondary color
    \begin{tabular}{llll}
        \rowcolor{\seccolor!50}
        ID & Name & Meilenstein & Schätzung in PT\\\bfhmidline
        T-1 & Klassifikation der Kugeln & M-1 & 6\\\bfhmidline
        T-2 & Aufsetzen Dokumentation & M-1 & 2\\\bfhmidline
        T-3 & Beschreibung Suchalgorithmus & M-1 & 3\\\bfhmidline
        T-4 & Implementation Suchalgorithmus & M-1 & 5\\\bfhmidline
        T-5 & Beschreibung der physikalischen Eigenschaften für die einfache Suche & M-1 & 6\\\bfhmidline
        T-6 & Implementation der einfachen Suche und deren Bewertungsfunktion& M-1 & 10\\\bfhmidline
        T-7 & Beschreibung der physikalischen Eigenschaften für die erweiterte Suche & M-2 & 6\\\bfhmidline
        T-8 & Implementation der erweiterten Suche und deren Bewertungsfunktion& M-2 & 8\\\bfhmidline
        T-9 & Überprüfen/Verbessern der Detektionsgenauigkeit & M-1 & 6\\\bfhmidline
        T-10 & Video erstellen & M-3 & 2\\\bfhmidline
        T-11 & Plakat schreiben & M-3 & 2\\\bfhmidline
        T-12 & Booklet-Eintrag schreiben & M-3 & 1\\\bfhmidline
        T-13 & Präsentation des Finaltags vorbereiten & M-3 & 2\\\bfhmidline
        T-14 & Präsentation der Verteidung vorbereiten & M-3 & 2\\\bfhmidline
        T-15 & Finalisieren Dokumentation andere Arbeiten & M-3 & 4\\\bfhmidline
        T-16 & Projektmanagement & Kein & 4\\\bfhmidline
        T-17 & Effizienz Erfassung und Steigerung der einfachen Suche & M-1 & 4\\\bfhmidline
        T-18 & Effizienz Erfassung und Steigerung der erweiterten Suche & M-2 & 2\\\bfhmidline
        T-19 & Dokumentation der Resultate der einfachen Suche & M-1 & 6\\\bfhmidline
        T-20 & Dokumentation der Resultate der erweiterten Suche & M-2 & 2\\\bfhmidline
        T-21 & Umbau in Unity & M-1 & 6\\\bfhmidline
        O-1 & Suche über mehrere Spielstände & M-2 & \\\bfhmidline
        O-2 & Detektion des Queues in 2D & M-2 & \\\bfhmidline
        O-3 & Detektion des Queues in 3D & M-2 & \\\bfhmidline
        O-4 & Stossberechnung anhand detektiertem Queue in 2D & M-2 & \\\bfhmidline
        O-5 & Stossberechnung anhand detektiertem Queue in 3D & M-2 & \\\bfhmidline
        O-6 & Spielerabhängige Heuristik & M-2 & \\\bfhmidline
        O-7 & Live-Verfolgung und Darstellung der Kugeln & M-2 & \\\bfhmidline
        \multicolumn{3}{c}{Total} & 90\\
    \end{tabular}
    \caption{Ziele}
    \label{tab:targets}
\end{table}

\subsubsection{Meilensteine}
Es werden drei Meilensteine definiert, welche auch aus optionalen Zielen bestehen können. Die Deadlines ergeben sich
aus den Schätzungen der zugewiesenen Arbeitspakete.

~\\

\textit{\textbf{Meilenstein 1 - 15.11.2021}}
Das Ziel ist eine sehr einfache simple Suche. Darunter zu verstehen ist eine Lösung, welche einen direkten Treffer findet
(Weiss -> Kugel -> Loch).\\
\textit{Code Deliverables:}
\begin{description}
    \item[Klassifikation - T-1]\hfill \\
    Alle Kugeln können entsprechend ihrere Farbe klassifiziert werden.
    \item[Suchalgorithmus für einfache Suche - T-4, T-6, T-17]\hfill \\
    Ein direkter Stoss wird in akzeptabler Zeit gefunden.
    \item[Unity-Umbau - T-21]\hfill \\
    Unity ist bereit für den Einsatz. Zum Umbau gehören insbesondere die Farbe der Markierung der Kugeln und deren
    Bahnen. Weiterhin muss Unity mehrere Suchergebnisse anzeigen können.
\end{description}
\textit{Dokumentation Deliverables:}
\begin{description}
    \item[Klassifikation - T-1]\hfill \\
    Das Vorgehen der Klassifikation wie deren Resultate und Genauigkeit sind dokumentiert.
    \item[Suchalgorithmus für Suche - T-3]\hfill \\
    Der Algorithmus der Suche ist theoretisch und mit Pseudocode beschrieben.
    Die theoretische Beschreibung muss nicht gänzlich mit der effektiven Implementation übereinstimmen, da diese auf
    Performance optimiert wird.
    \item[Resultate der einfachen Suche - T-19]\hfill \\
    In den Resultaten ist die Genauigkeit und Performance des einfachen Suchvorgangs
    beschrieben.
    \item[Physik der einfachen Suche - T-5]\hfill \\
    Die benötigte Physik der einfachen Suche ist beschrieben.
    \item[Bewertungsfunktion - T-6]\hfill \\
    Die Bewertungsfunktion der einfachen Suche ist dokumentiert.
\end{description}

~\\

\textit{\textbf{Meilenstein 2 - 13.12.2021}}
Das Ziel ist eine erweiterte Suche, die auch indirekte Stösse über weitere Kugeln oder Banden finden kann.
Optional sollen auch mehrere Stösse berücksichtigt werden.\\
\textit{Code Deliverables:}
\begin{description}
    \item[Suchalgorithmus für erweiterte Suche - T-8, T-18]\hfill \\
    Ein indirekter Stoss wird in akzeptabler Zeit gefunden.
    \item[Suchalgorithmus über mehrere Stösse - O-1]\hfill \\
    Es werden mehrere Spielstände bei der Suche berücksichtigt.
    \item[Queue in 2D detektieren - O-2]\hfill \\
    Der Queue wird als 2D-Objekt detektiert.
    \item[Queue in 3D detektieren - O-3]\hfill \\
    Der Queue wird mittels Tiefeninformationen der Kamera als 3D-Objekt detektiert.
    \item[Stossberechnung anhand detektiertem 2D-Queue - O-4]\hfill \\
    Der Stoss wird je nach Haltung des Queues in 2D berechnet. Es wird angenommen, dass der Queue zentral auf die
    weisse Kugel gerichtet ist.
    \item[Stossberechnung anhand detektiertem 3D-Queue - O-5]\hfill \\
    Der Stoss wird je nach Haltung des Queues in 3D berechnet. Der Queue muss nicht mehr zentral auf die weisse
    Kugel gerichtet sein.
    \item[Spielerabhängige Heuristik - O-6]\hfill \\
    Je nach Spieler kann eine andere Heuristik zur Bewertung der Stösse eingestellt werden. Durch die Unterscheidung
    können für professionelle Spieler erfolgsversprechendere schwerer durchzuführende und für Anfänger
    eher leichtere Stösse gefunden werden.
    \item[Live-Verfolgung und Darstellung der Kugeln - O-7]\hfill \\
    Die Kugeln werden ohne Benutzereingabe getrackt und deren Position über den Projektor dargestellt.
\end{description}
\textit{Dokumentation Deliverables:}
\begin{description}
    \item[Physik der erweiterten Suche - T-7]\hfill \\
    Die benötigte Physik der erweiterten Suche ist beschrieben.
    \item[Resultate der erweiterten Suche - T-20]\hfill \\
    In den Resultaten ist die Genauigkeit und Performance des erweiterten Suchvorgangs
    beschrieben.
    \item[Bewertungsfunktion - T-8]\hfill \\
    Die Bewertungsfunktion der erweiterten Suche ist dokumentiert.
\end{description}

~\\

\textit{\textbf{Meilenstein 3 - 17.01.2022}}
Das Ziel ist der Abschluss aller Arbeiten zu denen auch Plakat, Booklet oder Video gehören.\\
\textit{Deliverables:}
\begin{description}
    \item[Video - T-10]\hfill \\
    Das finale Video ist erstellt.
    \item[Plakat - T-11]\hfill \\
    Das Plakat ist erstellt.
    \item[Booklet - T-12]\hfill \\
    Der Booklet-Eintrag ist erstellt.
    \item[Präsentation für Finaltag - T-13]\hfill \\
    Die Präsentation/Ausstellung für den Finaltag ist vorbereitet.
    \item[Präsentation für Verteidigung - T-14]\hfill \\
    Die Präsentation für die Verteidigung ist vorbereitet.
    \item[Finalisieren der Arbeiten - T-15]\hfill \\
    Die Dokumentation wie auch der Code sind abgeschlossen.
\end{description}