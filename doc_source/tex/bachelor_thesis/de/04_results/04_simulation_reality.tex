\section{Vergleich Simulation und Realität}
Um die Realitätsnähe der Simulation zu quantifizieren, wird der Ansatz über die Analyse von Aufnahmen diverser real
durchgeführter Stösse verfolgt. Dabei ist die Geschwindigkeit, welche die Kugel zu Beginn durch den Queue erfährt,
unbekannt. Diese wird in einem ersten Schritt eruiert. Bekannt dabei ist die Zeit sowie die Distanz, welche über die
Selektion einzelner Frames berechnet werden. Die Zeit ergibt sich durch den Abstand zweier Frames. Bei einem Video mit
30 FPS liegt der Fehler bei $\frac{1}{30} t [s] = 0.03 [s]$. Die Distanz wird über zwei selektierte Pixelpositionen
auf denselben Frames zur Berechnung der Zeit bestimmt. Diese Positionen werden zu Modellkoordinaten $P_0, P_1$ übersetzt, um die
Distanz in der Einheit $[mm]$ anzugeben. Anschliessend kann über die Formel \ref{eq:simvsReal:initialVelocity} die Anfangsgeschwindigkeit berechnet
werden, wobei auch die Gleitreibung $\mu_g$ sowie die Rollreibung $\mu_r$ bekannt sein müssen\footnote{Die Herleitung
erfolgt in Kapitel \ref{anhang:herleitung:initialVelocityWithTime}.}.
\begin{align}
    v_0^2 \cdot \frac{2 \cdot (\mu_g - \mu_r)}{49 \cdot g \cdot \mu_g^2} + v_0 \cdot \frac{t \cdot (2 \cdot \mu_r + 5 \cdot \mu_g)}{7 \cdot \mu_g} - \frac{1}{2} \cdot g \cdot \mu_r \cdot t^2 - s = 0\label{eq:simvsReal:initialVelocity}
\end{align}

Es wird eine quadratische Formel gelöst, welche zwei Resultate $v^1_0, v^2_0$ liefert. Von diesen wird nur die
kleinste positive Lösung $v_0 = \min{(\max{(0, v^1_0)}, \max{(0, v^2_0)})}$ in Betracht gezogen.

Sobald $v_0$ bestimmt ist, kann der Geschwindigkeitsvektor $\vec{v_0}$ berechnet werden. Dies geschieht über den Einheitsvektor,
welcher über die beiden Punkte zur Bestimmung der Distanz gegeben ist:
\begin{align}
    \vec{d} = P_1 - P_0\\
    \hat{d} = \frac{\vec{d}}{\norm{\vec{d}}}\\
    \vec{v_0} = v_0 \cdot \hat{d}
\end{align}

In einem weiteren Schritt kann die Simulation mit der errechneten Realität durchgeführt werden. Es resultiert ein Simulationsmodell.
Dieses wird mit einem manuell erstellten Simulationsmodell abgeglichen, bei welchem alle Positionen der Kugeln und Ereigniszeitpunkte
anhand der Aufnahmen bestimmt wurden. Die Summe der Differenzen zwischen den erwarteten Zeiten und Positionen bilden den Gesamtfehler.