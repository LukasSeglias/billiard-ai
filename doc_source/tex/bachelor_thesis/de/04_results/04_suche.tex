\section{Suche}
Dieses Kapitel beinhaltet die Beschreibung und Diskussion der vorgeschlagenen Suchresultate. Diese werden aufgeteilt
in einzelne Funktionsumfänge, da die Interpretation ansonsten zu komplex wird. Die Funktionsumfänge beschreiben die Suche
über einen einzelnen Stoss, über die Möglichkeit einer oder mehrerer Bandeninteraktionen, die Berücksichtigung verschiedener
Startgeschwindigkeiten und der vorausschauende Blick in die Tiefe.
Ein Suchresultat wird wiederum aufgrund vier verschiedener
Kriterien beurteilt. So spielt die Distanz des zurückzulegenden Weges, der Kollisionswinkel einzelner Kugeln, die Nähe
zum Loch einer einzulochenden Kugel und die benötigte Startgeschwindigkeit eine Rolle.

In einem ersten Teil wird sich nur der Suche über einen Stoss ohne Bandeninteraktion und ohne
verfielfältigter Startgeschwindigkeit gewidmet. Der zweite Teil beinhaltet zusätzlich verschiedene Startgeschwindigkeiten.
Im dritten Teil werden verschiedene Startgeschwindigkeiten wiederum ignoriert, aber dafür Bandeninteraktionen berücksichtigt.
Der vierte Teil beinhaltet den kompletten Funktionsumfang.

Um die Resultate der Suche zu bewerten werden nachfolgend Situationen aufgestellt und die von der Suche
vorgeschlagenen Stösse untersucht auf Basis der Kriterien und berücksichtigten Funktionsumfänge untersucht.

\subsection{Die einfache Suche}\label{kap:suche:die_einfache_suche}
Die Situation zur einfachen Suche ist in Abbildung \ref{fig:search_situation_1} ersichtlich.
Der Spielball ist auf der linken Seite des Tisches und es liegen einige rote Kugeln auf dem Tisch verstreut.
Die Kugeln 1, 2 und 3 sind nahe bei den Löchern unten-links, oben-rechts resp. oben-zentral.
Kugel 4 ist in der Nähe des Spielballs, die Kugel 5 ist nahe dem Tischzenturm aufgestellt.

%
%State (for reproducing the results):
%WHITE1, WHITE, -564.147, 9.06096
%RED2, RED, -47.218, -100.06
%RED4, RED, 2.93182, 398.78
%RED5, RED, -865.047, -388.453
%RED6, RED, -629.728, 87.0048
%RED7, RED, 851.622, 398.78
%
\begin{figure}[h!]
    \begin{center}
        \includegraphics[width=0.4\linewidth]{../common/04_results/resources/simple_search/situation_diverse.PNG}
    \end{center}
    \caption{Situation 1: Einige verstreute Kugeln}
    \label{fig:search_situation_1}
\end{figure}

Nach Betrachten dieser Situation sind einige mögilche Stösse denkbar:
\begin{enumerate}
    \item Kugel 1 ins Loch unten-links.
    \item Kugel 2 ins Loch oben-rechts.
    \item Kugel 3 ins Loch oben-zentral.
    \item Kugel 4 ins Loch oben-links.
    \item Kugel 5 ins Loch unten-rechts.
    \item Kugel 5 ins Loch unten-zentral.
\end{enumerate}

In Abbildung \ref{fig:situation_1_solutions} sind die von der Suche gefundenen Stösse in aufsteigender Schwierigkeit abgebildet.
Mit den ersichtlichen weissen Linien auf jeder Abbildung ist der Pfad jeder Kugel, nicht nur des Spielballs, aufgezeichnet.
Diese voraussichtlichen Pfade werden nach der Suche mithilfe der Simulation des gefundenen Stosses eingezeichnet.

% TODO: bilder ersetzen, sobald die simulation final ist, damit der Pfad der weissen Kugel etwa stimmt.
\begin{figure}[h!]
    \centering
    \begin{subfigure}[b]{0.3\textwidth}
        \centering
        \includegraphics[width=1.0\linewidth]{../common/04_results/resources/simple_search/situation_diverse_solution_1.PNG}
        \caption{Stoss 1}
        \label{fig:situation_1_solution_1}
    \end{subfigure}
    \hfill
    \begin{subfigure}[b]{0.3\textwidth}
        \centering
        \includegraphics[width=1.0\linewidth]{../common/04_results/resources/simple_search/situation_diverse_solution_2.PNG}
        \caption{Stoss 2}
        \label{fig:situation_1_solution_2}
    \end{subfigure}
    \hfill
    \begin{subfigure}[b]{0.3\textwidth}
        \centering
        \includegraphics[width=1.0\linewidth]{../common/04_results/resources/simple_search/situation_diverse_solution_3.PNG}
        \caption{Stoss 3}
        \label{fig:situation_1_solution_3}
    \end{subfigure}
    \begin{subfigure}[b]{0.3\textwidth}
        \centering
        \includegraphics[width=1.0\linewidth]{../common/04_results/resources/simple_search/situation_diverse_solution_4.PNG}
        \caption{Stoss 4}
        \label{fig:situation_1_solution_4}
    \end{subfigure}
    \hfill
    \begin{subfigure}[b]{0.3\textwidth}
        \centering
        \includegraphics[width=1.0\linewidth]{../common/04_results/resources/simple_search/situation_diverse_solution_5.PNG}
        \caption{Stoss 5}
        \label{fig:situation_1_solution_5}
    \end{subfigure}
    \hfill
    \begin{subfigure}[b]{0.3\textwidth}
        \centering
        \includegraphics[width=1.0\linewidth]{../common/04_results/resources/simple_search/situation_diverse_solution_6.PNG}
        \caption{Stoss 6}
        \label{fig:situation_1_solution_6}
    \end{subfigure}
    \hfill
    \begin{subfigure}[b]{0.3\textwidth}
        \centering
        \includegraphics[width=1.0\linewidth]{../common/04_results/resources/simple_search/situation_diverse_solution_7.PNG}
        \caption{Stoss 7}
        \label{fig:situation_1_solution_7}
    \end{subfigure}
    \caption{Gefundene Stösse zu Situation 1 nach bewerteter Schwierigkeit aufsteigend}
    \label{fig:situation_1_solutions}
\end{figure}

Die zuvor beschriebenen denkbaren Stösse wurden tatsächlich gefunden.
Der Stoss 1 wurde als der einfachste Stoss bewertet, weil die gesamte zurückzulegende Distanz klein ist,
der Pfad der beteiligten Kugeln insgesamt sehr geradlinig ist und weil die Kugel 1 sehr nahe am Loch unten-links liegt.
Der zweiteinfachste Stoss 2 zeigt eine grössere Distanz, welche der Spielball zurücklegen muss und daher erfordert
der Stoss auch eine höhere Startgeschwindigkeit des Spielballs wodurch dieser als schwieriger eingestuft wird als Stoss 1.
Dafür ist die Kugel 2 sehr nahe am Loch oben-rechts, wodurch die Kugel nicht mit absoluter Genauigkeit getroffen werden
muss, damit diese den erwünschten Pfad nicht so weit verlässt, dass sie das Ziel verfehlt.

Im Stoss 3 ist die Gesamtdistanz klein, allerdings ist die Distanz der Kugel 4 zum Spielball kleiner als die Distanz zum Loch.
Dadurch führt ein kleiner Fehler beim Treffen der Kugel zu einer grösseren Abweichung des Pfades.
In diesem Beispiel könnte dieser Fehler allerdings relativ klein sein, da der Pfad geradlinig ist und der Spielball die Kugel
nicht in einem Winkel treffen muss.

Stoss 2 wurde besser als Stoss 3 bewertet, weil bei Stoss 2 die Kugel näher am Loch liegt als bei Stoss 3. Die anderen
Bewertungskriterien Distanz und Winkel sind bei Stoss 3 besser als bei Stoss 2.
Ob diese Rangfolge objektiv korrekt ist, ist schwierig zu beurteilen.

Der Stoss 5 wird als schwieriger bewertet, weil der Winkel, in dem der Spielball die Kugel 3 treffen muss ca 45$^{\circ}$ ist.
Im Gegensatz dazu ist die Distanz der Kugel zum Loch sehr klein und die Gesamtdistanz ist relativ klein, was diesen Stoss
wiederum einfacher macht. Trotzdem hat in diesem Fall der Winkel die Bewertung so stark reduziert, dass Stoss 4 trotzdem
als einfacher eingestuft wird.

Der Stoss 6 versenkt die Kugel 2 indirekt über die Kugel 5. Dieser Stoss wird dennoch einfacher bewertet als der
nachfolgende Stoss 7, der keine Indirektion beinhaltet, da der Auftrittswinkel nicht so gross ist und die weisse Kugel
nicht so stark angespielt werden muss.

Stoss 7 zeigt die Idee, Kugel 5 ins Loch unten-zentral zu spielen.
Dazu muss diese in einem sehr grossen Winkel angespielt werden, was die Treffgenauigkeit reduziert und eine
erhöhte Startgeschwindigkeit des Spielballs erfordert.

Aus den obigen Resultaten ist ersichtlich, dass die Suche diejenigen Stösse vorschlägt, welche nach Studium des Spielstandes
möglich erscheinen. Zudem berücksichtigt die Bewertung der Schwierigkeit eines Stosses die gesamte Distanz, die Winkel, die Nähe
der einzulochenden Kugel zum Loch und die erforderliche Startgeschwindigkeit.
\newpage

\subsection{Die einfache Suche - mit verfielfältigter Startgeschwindigkeit}
Es wird wiederum die Situation wie in Kapitel \ref{kap:suche:die_einfache_suche} betrachtet. Diesmal wird aber die Verfielfältigung der Startgeschwindigkeiten
zugelassen, insgesamt werden jeweils drei verschieden starke Stargeschwindigkeiten berücksichtigt.
Als Resultat wird in etwa dieselbe Abfolge der Suchresultate wie in Kapitel \ref{kap:suche:die_einfache_suche} erwartet, es kann aber durchaus
sein, dass gewisse Stösse mit geringer Distanz und kleinem Winkel durch die hohe Startgeschwindigkeit schlechter bewertet werden
als solche mit grösserer Distanz und grösserem Winkel.

In Abbildung \ref{fig:situation_1_solutions_startgeschwindigkeit} sind die von der Suche gefundenen Stösse in aufsteigender Schwierigkeit abgebildet.
Insgesamt wurden 21 Lösungen gefunden, davon werden nur die zehn besten diskutiert.

% TODO: bilder ersetzen, sobald die simulation final ist, damit der Pfad der weissen Kugel etwa stimmt.
\begin{figure}[h!]
    \centering
    \begin{subfigure}[b]{0.3\textwidth}
        \centering
        \includegraphics[width=1.0\linewidth]{../common/04_results/resources/simple_search/situation_diverse_solution_velocity_1.PNG}
        \caption{Stoss 1}
        \label{fig:situation_velocity_1_solution_1}
    \end{subfigure}
    \hfill
    \begin{subfigure}[b]{0.3\textwidth}
        \centering
        \includegraphics[width=1.0\linewidth]{../common/04_results/resources/simple_search/situation_diverse_solution_velocity_2.PNG}
        \caption{Stoss 2}
        \label{fig:situation_velocity_1_solution_2}
    \end{subfigure}
    \hfill
    \begin{subfigure}[b]{0.3\textwidth}
        \centering
        \includegraphics[width=1.0\linewidth]{../common/04_results/resources/simple_search/situation_diverse_solution_velocity_3.PNG}
        \caption{Stoss 3}
        \label{fig:situation_velocity_1_solution_3}
    \end{subfigure}
    \hfill
    \begin{subfigure}[b]{0.3\textwidth}
        \centering
        \includegraphics[width=1.0\linewidth]{../common/04_results/resources/simple_search/situation_diverse_solution_velocity_4.PNG}
        \caption{Stoss 4}
        \label{fig:situation_velocity_1_solution_4}
    \end{subfigure}
    \hfill
    \begin{subfigure}[b]{0.3\textwidth}
        \centering
        \includegraphics[width=1.0\linewidth]{../common/04_results/resources/simple_search/situation_diverse_solution_velocity_5.PNG}
        \caption{Stoss 5}
        \label{fig:situation_velocity_1_solution_5}
    \end{subfigure}
    \hfill
    \begin{subfigure}[b]{0.3\textwidth}
        \centering
        \includegraphics[width=1.0\linewidth]{../common/04_results/resources/simple_search/situation_diverse_solution_velocity_6.PNG}
        \caption{Stoss 6}
        \label{fig:situation_velocity_1_solution_6}
    \end{subfigure}
    \hfill
    \begin{subfigure}[b]{0.3\textwidth}
        \centering
        \includegraphics[width=1.0\linewidth]{../common/04_results/resources/simple_search/situation_diverse_solution_velocity_7.PNG}
        \caption{Stoss 7}
        \label{fig:situation_velocity_1_solution_7}
    \end{subfigure}
    \hfill
    \begin{subfigure}[b]{0.3\textwidth}
        \centering
        \includegraphics[width=1.0\linewidth]{../common/04_results/resources/simple_search/situation_diverse_solution_velocity_8.PNG}
        \caption{Stoss 8}
        \label{fig:situation_velocity_1_solution_8}
    \end{subfigure}
    \hfill
    \begin{subfigure}[b]{0.3\textwidth}
        \centering
        \includegraphics[width=1.0\linewidth]{../common/04_results/resources/simple_search/situation_diverse_solution_velocity_9.PNG}
        \caption{Stoss 9}
        \label{fig:situation_velocity_1_solution_9}
    \end{subfigure}
    \hfill
    \begin{subfigure}[b]{0.3\textwidth}
        \centering
        \includegraphics[width=1.0\linewidth]{../common/04_results/resources/simple_search/situation_diverse_solution_velocity_10.PNG}
        \caption{Stoss 10}
        \label{fig:situation_velocity_1_solution_10}
    \end{subfigure}
    \caption{Gefundene Stösse zu Situation 1 nach bewerteter Schwierigkeit aufsteigend mit verschiedenen Startgeschwindigkeiten}
    \label{fig:situation_1_solutions_startgeschwindigkeit}
\end{figure}

Wie zu erwarten sieht die Abfolge der Suchresultate nicht wesentlich anders aus wie vorhin. Was auffällt ist die Tatsache,
dass eine erhöhte Geschwindigkeit durchaus dazu führen kann, dass ein Stoss, der einen grösseren Weg und/oder Winkel aufweist,
besser gewertet werden kann. Dies ist z.B. bei Stoss 3 ersichtlich, der besser gewertet wird als Stoss 4 oder bei
Stoss 8, der schwieriger als Stoss 6 und 7 ist. Bei Stoss 8 könnte argumentiert werden, dass die Stärke
in dem Fall keinen grossen Unterschied macht, da die Kugel 2 mit grosser Wahrscheinlichkeit eingelocht werden kann,
jedoch ist generell ein sanftes Billardspiel zu bevorzugen, weswegen die Bewertung in dem Sinne durchaus gerechtfertigt
sein kann.

\newpage

\subsection{Die einfache Suche - mit Bandenkollisionen}
In diesem Fall werden verschiedene Startgeschwindigkeiten nicht berücksichtigt, jedoch wird das Bandenspiel zugelassen.
Auch hier sollten sich die Resultate nicht wesentlich zur einfachen Suche in Kapitel \ref{kap:suche:die_einfache_suche} ändern.
Ein Unterschied ist vor allem gegen Ende bei Stoss 6 und 7 zu erwarten, so wird eventuell eher ein Spiel über eine Bande berücksichtigt,
als der starke Stoss 7 oder der indirekte Stoss 6.

In Abbildung \ref{fig:situation_1_solutions_bande} sind die Resultate in aufsteigender Schwierigkeit gegeben.

\begin{figure}[h!]
    \centering
    \begin{subfigure}[b]{0.3\textwidth}
        \centering
        \includegraphics[width=1.0\linewidth]{../common/04_results/resources/simple_search/situation_diverse_solution_rail_1.PNG}
        \caption{Stoss 1}
        \label{fig:situation_rail_1_solution_1}
    \end{subfigure}
    \hfill
    \begin{subfigure}[b]{0.3\textwidth}
        \centering
        \includegraphics[width=1.0\linewidth]{../common/04_results/resources/simple_search/situation_diverse_solution_rail_2.PNG}
        \caption{Stoss 2}
        \label{fig:situation_rail_1_solution_2}
    \end{subfigure}
    \hfill
    \begin{subfigure}[b]{0.3\textwidth}
        \centering
        \includegraphics[width=1.0\linewidth]{../common/04_results/resources/simple_search/situation_diverse_solution_rail_3.PNG}
        \caption{Stoss 3}
        \label{fig:situation_rail_1_solution_3}
    \end{subfigure}
    \hfill
    \begin{subfigure}[b]{0.3\textwidth}
        \centering
        \includegraphics[width=1.0\linewidth]{../common/04_results/resources/simple_search/situation_diverse_solution_rail_4.PNG}
        \caption{Stoss 4}
        \label{fig:situation_rail_1_solution_4}
    \end{subfigure}
    \hfill
    \begin{subfigure}[b]{0.3\textwidth}
        \centering
        \includegraphics[width=1.0\linewidth]{../common/04_results/resources/simple_search/situation_diverse_solution_rail_5.PNG}
        \caption{Stoss 5}
        \label{fig:situation_rail_1_solution_5}
    \end{subfigure}
    \hfill
    \begin{subfigure}[b]{0.3\textwidth}
        \centering
        \includegraphics[width=1.0\linewidth]{../common/04_results/resources/simple_search/situation_diverse_solution_rail_6.PNG}
        \caption{Stoss 6}
        \label{fig:situation_rail_1_solution_6}
    \end{subfigure}
    \hfill
    \begin{subfigure}[b]{0.3\textwidth}
        \centering
        \includegraphics[width=1.0\linewidth]{../common/04_results/resources/simple_search/situation_diverse_solution_rail_7.PNG}
        \caption{Stoss 7}
        \label{fig:situation_rail_1_solution_7}
    \end{subfigure}
    \hfill
    \begin{subfigure}[b]{0.3\textwidth}
        \centering
        \includegraphics[width=1.0\linewidth]{../common/04_results/resources/simple_search/situation_diverse_solution_rail_8.PNG}
        \caption{Stoss 8}
        \label{fig:situation_rail_1_solution_8}
    \end{subfigure}
    \hfill
    \begin{subfigure}[b]{0.3\textwidth}
        \centering
        \includegraphics[width=1.0\linewidth]{../common/04_results/resources/simple_search/situation_diverse_solution_rail_9.PNG}
        \caption{Stoss 9}
        \label{fig:situation_rail_1_solution_9}
    \end{subfigure}
    \hfill
    \begin{subfigure}[b]{0.3\textwidth}
        \centering
        \includegraphics[width=1.0\linewidth]{../common/04_results/resources/simple_search/situation_diverse_solution_rail_10.PNG}
        \caption{Stoss 10}
        \label{fig:situation_rail_1_solution_10}
    \end{subfigure}
    \caption{Gefundene Stösse zu Situation 1 nach bewerteter Schwierigkeit aufsteigend mit Berücksichtigung der Banden}
    \label{fig:situation_1_solutions_bande}
\end{figure}

Bis zu Stoss 5 gibt es keine Änderungen der Vorschläge. Bei Stoss 6 wird
anstelle des indirekten Stoss auf Kugel 2 in Kapitel \ref{kap:suche:die_einfache_suche} jedoch eine Lösung über
die untere Bande auf Kugel 3 empfohlen. Dies macht durchaus Sinn, da das Spiel über
die untere Bande einerseits direkter ist und andererseits eine ähnliche Distanz
mit kleineren Winkeln vorliegt. Zudem liegt die Kugel 3 näher am Loch als die Kugel 2.

Stoss 7 schlägt eine ähnliche Lösung wie Stoss 6 vor, jedoch über die obere Bande auf die Kugel 5, die im
mittleren unteren Loch versenkt wird. Auch dies ist eine direktere Lösung mit kleineren Distanzen
als der Stoss 6 aus Kapitel \ref{kap:suche:die_einfache_suche}, er wird aber auch schlechter als Stoss 6 dieses Kapitels
gewertet, da die Kugel 5 einen längeren Weg zum Loch zurücklegen muss, was die Wahrscheinlichkeit
eines akkumulierten Fehlers erhöht.

Stoss 8 entspricht nun dem Stoss 6 aus Kapitel \ref{kap:suche:die_einfache_suche}. Dieser reiht sich vor
dem Stoss 9 dieses Kapitels ein, wo es ebenfalls um das Versenken der Kugel 2 geht. Da die Distanzen
zwischen den Kugeln kleiner sind und die Winkel ungefähr gleich gross, wird der indirekte Stoss
über eine Kugel bevorzugt.

Abschliessend zeigt Stoss 10 das direkte Versenken der Kugel 5 im mittleren unteren Loch.
Da der Winkel und die Startgeschwindigkeit sehr gross sind, ist das Resultat durchaus nachzuvollziehen.

\newpage
% TODO: resultate der suche über mehrere stösse beschreiben, sofern umgesetzt

% TODO: maybe describe, otherwise delete images
%
%State (for reproducing the results):
%WHITE0, WHITE, -0.925876, 9.06096
%RED1, RED, -0.925876, -345.584
%RED2, RED, -0.925876, 246.79
%

%\begin{figure}[h!]
%    \begin{center}
%        \includegraphics[width=0.4\linewidth]{../common/04_results/resources/simple_search/situation_similar_distance_but_one_is_closer_to_pocket.PNG}
%    \end{center}
%    \caption{Situation 2: Eine Kugel ist näher am Ziel als eine andere}
%    \label{fig:search_situation_2}
%\end{figure}

%\begin{figure}
%    \centering
%    \begin{subfigure}[b]{0.4\textwidth}
%        \centering
%        \includegraphics[width=1.0\linewidth]{../common/04_results/resources/simple_search/situation_similar_distance_but_one_is_closer_to_pocket_solution_1.PNG}
%        \caption{Stoss 1}
%        \label{fig:situation_2_solution_1}
%    \end{subfigure}
%    \hfill
%    \begin{subfigure}[b]{0.4\textwidth}
%        \centering
%        \includegraphics[width=1.0\linewidth]{../common/04_results/resources/simple_search/situation_similar_distance_but_one_is_closer_to_pocket_solution_2.PNG}
%        \caption{Stoss 2}
%        \label{fig:situation_2_solution_2}
%    \end{subfigure}
%    \caption{Gefundene Stösse zu Situation 2 nach bewerteter Schwierigkeit aufsteigend}
%    \label{fig:situation_2_solutions}
%\end{figure}

% TODO: maybe describe, otherwise delete images
%
%State (for reproducing the results):
%WHITE0, WHITE, 14.5049, 9.06096
%RED1, RED, -598.867, -146.827
%RED2, RED, 492.858, -240.359
%
%\begin{figure}[h!]
%    \begin{center}
%        \includegraphics[width=0.4\linewidth]{../common/04_results/resources/simple_search/situation_similar_distance_different_angles.PNG}
%    \end{center}
%    \caption{Situation 3: Eine Kugel muss in einem grösseren Winkel angespielt werden als die andere.}
%    \label{fig:search_situation_3}
%\end{figure}

%\begin{figure}
%    \centering
%    \begin{subfigure}[b]{0.4\textwidth}
%        \centering
%        \includegraphics[width=1.0\linewidth]{../common/04_results/resources/simple_search/situation_similar_distance_different_angles_solution_1.PNG}
%        \caption{Stoss 1}
%        \label{fig:situation_3_solution_1}
%    \end{subfigure}
%    \hfill
%    \begin{subfigure}[b]{0.4\textwidth}
%        \centering
%        \includegraphics[width=1.0\linewidth]{../common/04_results/resources/simple_search/situation_similar_distance_different_angles_solution_2.PNG}
%        \caption{Stoss 2}
%        \label{fig:situation_3_solution_2}
%    \end{subfigure}
%    \caption{Gefundene Stösse zu Situation 3 nach bewerteter Schwierigkeit aufsteigend}
%    \label{fig:situation_3_solutions}
%\end{figure}

