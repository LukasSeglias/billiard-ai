\chapter{Zusammenfassung}\label{kap:summary}
Billard ist eine Sportart, die strategisches Denken, ein gewisses Verständnis von Physik und eine ruhige Hand erfordert.
Für Anfänger kann es schwierig sein, eine Spielsituation zu analysieren, geeignete Stösse zu finden, und den passenden auszuwählen.
Zudem muss die Stärke, mit der die weisse Kugel angestossen werden soll, abgeschätzt werden.
Was nach dem Stoss passiert, muss sich der Spieler aufgrund seiner Erfahrung im Spiel vorstellen.

Mit Billiard-AI wurde ein System entwickelt, das zum Ziel hat, den Spieler zu unterstützen und ihm wichtige Erkenntnisse
in leicht verständlicher Art zugänglich zu machen.
Die in dieser Arbeit betrachtete Spielart ist Snooker, da diese die Weitsicht des Spielers erfordert, über mehrere Stösse zu planen.

Es wurde ein physischer Aufbau mit einem Billardtisch, einer Kamera und einem Projektor verwendet.
Über die Bilder der Kamera werden die Positionen und Farben aller Kugeln kontinuierlich und in Echtzeit erkannt.
Mithilfe des Projektors werden dem Spieler Hilfestellungen und Informationen direkt auf dem Tisch eingeblendet.

Um dem Spieler bei der Wahl des Stosses zu helfen, wird die erkannte Spielsituation analysiert und mögliche Stösse gefunden.
Diese werden nach verschiedenen, geeigneten Kriterien bewertet, um deren Schwierigkeit einzuschätzen und dem Spieler
einfache Stösse vorzuschlagen.
Es werden sowohl direkte wie auch Stösse über die Banden gefunden und berücksichtigt.

Das System ist fähig, die benötigte Geschwindigkeit eines Stosses zu bestimmen, damit die gewünschte Kugel eingelocht wird.
Gefundene Stösse werden mit unterschiedlichen Startgeschwindigkeiten physikalisch simuliert, um die Auswirkungen
des Stosses auf die Spielsituation zu untersuchen und dem Spieler diese animiert darzustellen.
Der Weg jeder beteiligten Kugel wird anhand von Linien auf dem Tisch angezeigt und durch die Animation projizierter Kugeln
ist der Ablauf des Stosses vom Anfang bis zum Schluss nachvollziehbar.
So sieht der Spieler die Konsequenzen des Stosses und kann den Billardqueue (Billardstock) an den angezeigten Linien ausrichten
und versuchen, die animierte Geschwindigkeit in den Stoss einfliessen zu lassen, damit dieser gelingt.

Das System kann das Wissen über die Spielsituation nach einem Stoss nutzen, um über mehrere Spielzüge zu planen.
Die Konsequenz ist, dass eine Abfolge sukzessiver Stösse dem Spieler präsentiert werden, welche dieser anschliessend
ausführen kann.
Damit kann der Spieler sein strategisches Denken und die korrekte Ausführung von Stössen mit Unterstützung üben.

Es wurde ein Spielmodus implementiert, in dem der Spieler nur durch das Billardspiel mit dem System interagiert.
Dazu wird dem Spieler ein Stoss vorgeschlagen und angezeigt.
Der Spieler kann diesen anschliessend mehr oder weniger erfolgreich ausführen.
Nachdem alle Kugeln wieder zum Stillstand gekommen sind, wird aufgrund dieser neuen Ausgangslage ein neuer Vorschlag berechnet.
Durch diese Interaktion kann der Spieler unterbrechungsfrei Billard spielen und trotzdem vom System profitieren.

Im Vergleich zu verwandten Arbeiten besitzt diese Arbeit einige Alleinstellungsmerkmale.
Dank dem physischen Aufbau ist die Arbeit nicht nur theoretischer, sondern auch praktischer Natur.

Die gesetzten Ziele wurden übertroffen und das Resultat ist eine \emph{Augmented Reality} Anwendung,
welche mithilfe von künstlicher Intelligenz Billardanfänger optimal unterstützt.
