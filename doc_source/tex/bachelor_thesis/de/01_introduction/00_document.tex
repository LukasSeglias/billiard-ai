\chapter{Einführung}
Wie viele andere Sportarten ist auch das Erlernen des Billardspiels eine schwierige Sache.
Als Spieler muss eine gegebene Spielsituation untersucht, ein möglicher und den eigenen Fähigkeiten entsprechender Stoss
gefunden und korrekt ausgeführt werden.
Es stellen sich Fragen wie \glqq Welche Kugel soll man anspielen?\grqq{},
\glqq Wie soll man die Kugel anspielen?\grqq{} oder \glqq Wie hält man den Queue richtig?\grqq{}.
Darüber hinaus gibt es noch weitere Überlegungen, welche den Profi vom Anfänger unterscheiden.
Ähnlich wie im Schach ist Weitsicht gefragt, um ein optimales Spiel zu verfolgen.
Es geht also nicht nur darum, eine Kugel zu versenken, sondern auch den Spielstand so zu verändern,
dass optimal weitergespielt werden kann.
Das Stichwort ist im Billard die sogenannte Platzierung der weissen Kugel.

Diese Arbeit widmet sich einem System, das den Spieler bei der Einführung in die schwierige Kunst des Billardspiels unterstützen soll.
Das in dieser Arbeit gewählte Spiel ist Snooker, da dieses eine sehr strategische Form des Billards ist und professionell ausgetragen wird.

Um eine Hilfestellung zu bieten, muss das System wissen, wie die Situation auf dem Billardtisch aussieht.
Dazu wurde ein Tisch mit einer Kamera und einem Projektor ausgestattet, um den aktuellen Spielstand in Echtzeit
zu erkennen und dem Spieler Hilfestellungen über den Projektor direkt auf den Tisch zu projizieren.
Der Spieler soll mit dem System interagieren können und direkte Reaktionen erhalten.

Das Gesamtsystem hat verschiedene Komponenten, zuerst müssen die Billardkugeln auf dem Tisch mittels der Kamera
aufgenommen und deren Position durch Bildverarbeitung und -analyse erkannt werden.
Des Weiteren muss die Farbe jeder Kugel erkannt werden, da dies für Snooker aufgrund der Spielregeln besonders relevant ist.
Diese Detektion muss eine hohe Genauigkeit und Robustheit aufweisen, um ein erhöhtes Vertrauen in das System zu erreichen.

Darauf aufbauend geht es um die Fragestellung, welchen Stoss der Spieler ausführen soll.
Dazu muss die Vielzahl an Möglichkeiten, die eine Spielsituation bietet, auf geeignete Weise durchsucht und die gefundenen
Stösse nach Schwierigkeit bewertet werden, um dem Spieler einen idealen Vorschlag zu machen.
Es sollen sowohl Stösse, welche eine Kugel direkt einlochen, als auch Stösse über die Banden berücksichtigt werden.

Ausserdem ist für die genannte Platzierung, die im Billard wichtig ist, eine Simulation von Billardstössen notwendig,
um die Situation zu berechnen, welche nach einem Stoss entsteht.
Aus dieser neuen Ausgangssituation können geeignete nächste Stösse gesucht und bewertet werden, um in der Wahl des
ersten Stosses mögliche zukünftige Stösse zu berücksichtigen.

Der Spieler soll nicht nur bei der Wahl, sondern auch bei der Ausführung von Stössen unterstützt werden.
Es muss klar kommuniziert werden, was das Ziel des Stosses ist, in welchem Winkel und mit welcher Geschwindigkeit
der Spielball getroffen werden muss und was die Auswirkungen auf den Spielstand sind.
Die Anzeige muss direkt auf dem Tisch erfolgen.

Es ist erwünscht, dass der Spieler auf natürliche Weise mit dem System interagiert, ohne das Spiel regelmässig zu unterbrechen.
Dazu soll automatisch erkannt werden, wann ein neuer Vorschlag gemacht werden muss, ohne den Spieler bei der Ausführung
eines Stosses zu unterbrechen.

In Kapitel \ref{kap:ziele} werden die Ziele dieser Arbeit näher beschrieben.
Kapitel \ref{billardAi} beschreibt die Implementation des Systems, deren Resultate in Kapitel \ref{kap:resultate}
diskutiert werden.
Es wird ein Ausblick in mögliche weitere Arbeiten am System in Kapitel \ref{kap:weitere_arbeiten} formuliert und
in Kapitel \ref{kap:fazit} enthält das Fazit dieser Arbeit.
