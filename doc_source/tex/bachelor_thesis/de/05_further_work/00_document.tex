\chapter{Weitere Arbeiten}\label{kap:weitere_arbeiten}
Aufbauend auf den in dieser Arbeit erzielten Resultaten sind potenzielle weitere Arbeiten für Erweiterungen und Verbesserungen
möglich.
Nachfolgend werden einige Ideen beschrieben.

\begin{description}
    \item[Vorwärtssuche]\mbox{} \\
    Die Suche nach Stössen wird, wie in Kapitel \ref{sec:kandidatensuche} beschrieben, ausgehend von den Löchern gestartet,
    wodurch diese als Rückwärtssuche bezeichnet werden kann.
    Der Raum der möglichen Stösse wird diskretisiert, weil nicht alle Möglichkeiten untersucht werden.
    Eine Vorwärtssuche würde viele Simulationen mit unterschiedlichen Anfangsbedingungen (Stossrichtung \& -stärke)
    erfordern und so eine hohe Anzahl möglicher Stösse erzielen.
    Dabei müsste der unendliche Raum der Anfangsbedingungen sinnvoll diskretisiert werden, um eine solche Suche zu ermöglichen.
    Wegen der Diskretisierung der Rückwärtssuche kann es in gewissen Situationen Stösse geben, welche nicht gefunden werden.
    Durch eine Kombination von Rückwärts- und Vorwärtssuche könnten mehr potenzielle Stösse untersucht werden.
    \item[3D-Simulation mit Spin]\mbox{} \\
    Die in Kapitel \ref{kap:simulation} beschriebene Simulation ist zweidimensional.
    Zur korrekten Behandlung aller Arten von Spin, beispielsweise Sidespin, müsste diese auf eine 3D-Simulation erweitert werden.
    Die Suche könnte weiterhin zweidimensional bleiben, würde dadurch aber auch keine Stösse mit Spin vorschlagen.
    Dabei könnte evtl. die bereits erwähnte Vereinigung mit einer Vorwärtssuche helfen, welche dem Stoss als
    Anfangsbedingung zusätzlich auch Spin geben könnte.
    \item[Verbesserter Ausfallswinkel]\mbox{} \\
    Die Berechnung des Ausfallswinkels nach einer Bandenkollision bei starken Stössen wie in
    Kapitel \ref{kandidatensuche:bandenkollisionstheorie} beschrieben, könnte verbessert werden, indem der Winkel entsprechend
    der Geschwindigkeit angepasst würde. Dazu müsste experimentell ein Wertebereich eruiert werden, für den der Ausfalls- dem
    Einfallswinkel entspricht. Die Simulation kann dann relativ einfach den Ausfallswinkel aufgrund der Geschwindigkeit
    berechnen. Die Suche liesse ein mögliches Resultat nur zu, wenn die Eintrittsgeschwindigkeit der Kugel an die Bande
    in diesem Wertebereich liegt, ansonsten gälte es als erfolgslos, was wiederum dazu führen würde,
    dass weniger Suchresultate, dafür aber qualitativ hochwertigere gefunden würden.
    \item[Kugelklassifikation]\mbox{} \\
    Die Genauigkeit der Klassifikation der Kugelfarben könnte weiter optimiert werden, um bessere Resultate zu erzielen.
    Dies ist bei den über den gesamten Tisch unterschiedlichen Lichtverhältnissen und projizierten Augmentationen herausfordernd.
    \item[Verbesserte Analyse]\mbox{} \\
    Der Einsatz einer Hochgeschwindigkeitskamera könnte nachgelagerte Analysen eines Spiels,
    dessen Aufzeichnung und spätere Wiedergabe erlauben.
    Dies würde vermutlich weitere Optimierungen in der Detektion erfordern.
    \item[Erweiterung auf Pool-Billard]\mbox{} \\
    Eine Ausweitung auf andere Arten des Billards, wie etwa Poolbillard, ist durch den Aufbau des Systems möglich, da
    Snooker-spezifische Komponenten von Allgemeingültigen getrennt wurden.
    Es müssten eine Detektion und Klassifikation für Poolbillard-Kugeln und einige Spielregeln implementiert werden.
    \item[Hand und Arm detektieren]\mbox{} \\
    Die Detektion von Händen und Armen des Spielers im Bild, würde es erlauben,
    die Detektionsfehler aus Abschnitt \ref{kap:detektion_arme_haende} zu vermeiden, wodurch das Spielerlebnis störungsfreier würde.
    \item[Queuerichtung detektieren]\mbox{} \\
    Des Weiteren könnte es für den Spieler von Nutzen sein, wenn das System die Position und Orientierung des Billardqueues
    erkennen könnte.
    Dank dieser Informationen könnte dem Spieler angezeigt werden, wie der Spielball bei einem Stoss in angezeigter Richtung des Queues
    an den Banden reflektieren würde, um so Erfahrungen mit Stössen über die Banden zu sammeln.
    Es gibt bereits verwandte Arbeiten, wo dies umgesetzt ist, siehe Kapitel \ref{kap:verwandte_arbeiten}.
    Ausserdem wäre es möglich, verschieden starke Stösse aufgrund der aktuellen Position und Orientierung des Queues zu simulieren
    und dem Spieler den besten Stoss animiert anzuzeigen.
    \item[Regelverstösse detektieren]\mbox{} \\
    Für ein erweitertes Spielerlebnis sind weitere Spielmodi denkbar, beispielsweise die Erkennung und Behandlung von
    Regelverstössen, dem Wechsel zwischen zwei Spielern und dem Zählen der Punktzahl jedes Spielers.
    In einem Trainingsmodus könnte dem Spieler ein Stoss angezeigt, dessen Ausführung beobachtet und dem Spieler
    Rückmeldung gegeben werden.
    Diese Idee entstand im Gespräch mit einem Snooker-Experten, welcher schon an Weltmeisterschaften teilgenommen hat, siehe Kapitel \ref{anhang:snooker-experte-treffen}.
    \item[Mehr Feedback an den Spieler]\mbox{} \\
    Über die Darstellung würde es sich anbieten, verschiedene Ereignisse mit speziellen Animationen oder über Ton hervorzuheben,
    beispielsweise wenn ein Spieler eine Kugel erfolgreich einlocht, oder wenn der Spielball ebenfalls versenkt wird.
    \item[Physikalische Parameter optimieren]\mbox{} \\
    Die Simulation und die Berechnung eines Lösungskandidaten erfordern einige bekannte Eigenschaften des Billardtisches und
    der Kugeln wie z.B. den Gleit- oder Rollreibungskoeffizienten, verschiedene Energieverlustfaktoren oder ein variables
    Bandenverhalten. Diese Parameter zu finden ist äusserst schwierig und aufwendig. Es wäre denkbar, dass einige repräsentative
    Stösse aufgezeichnet und wichtige Schlüsseleigenschaften wie Ereigniszeitpunkt, Position und Richtung annotiert werden,
    wie es in dieser Arbeit für den Simulations- Realitätsabgleich bereits vorgenommen
    wurde (Siehe Kapitel \ref{kap:vergleich_simulation_und_realitaet}). Anhand dieser könnte über ein Optimierungsverfahren
    die Differenz zwischen der Simulation und Realität minimiert werden, indem die genannten Eigenschaften des Tisches und
    der Kugeln als Variablen behandelt und im Zuge der Optimierung angepasst werden. Dies ergäbe eine genauere Eichung auf
    die zugrundeliegenden Stösse.
    \item[Detektion, Simulation und Suche über Machine Learning lernen]\mbox{} \\
    Es wäre denkbar, mithilfe von Machine Learning die Detektion, Simulation und Stosssuche zu implementieren.
    Dazu müsste eine grosse Datensammlung angelegt und annotiert werden.
    Dies hat das Potential die Genauigkeit dieser Komponenten zu verbessern, würde allerdings einen sehr grossen Aufwand bedeuten.
\end{description}
