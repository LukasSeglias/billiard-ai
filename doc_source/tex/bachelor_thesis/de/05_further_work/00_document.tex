\chapter{Weitere Arbeiten}\label{kap:weitere_arbeiten}
Auf den in dieser Arbeit erzielten Resultaten sind potenzielle weitere Arbeiten für Erweiterungen und Verbesserungen
möglich.
Nachfolgend werden einige Ideen beschrieben.

Die Suche nach Stössen wird, wie in Kapitel \ref{sec:kandidatensuche} beschrieben ausgehend von den Löchern gestartet,
wodurch diese als Rückwärtssuche bezeichnet werden kann.
Der Raum der möglichen Stösse wird diskretisiert, weil nicht allen Möglichkeiten untersucht werden.
Eine Vorwärtssuche würde viele Simulationen mit unterschiedlichen Anfangsbedingungen (Stossrichtung \& -stärke)
durchführen und so mögliche Stösse finden.
Dabei müsste der unendliche Raum der Anfangsbedingungen sinnvoll diskretisiert werden, um eine solche Suche zu ermöglichen.
Wegen der Diskretisierung der Rückwärtssuche kann es in gewissen Situationen Stösse geben, welche nicht gefunden werden.
Durch eine Kombination von Rückwärts- und Vorwärtssuche könnten mehr potenzielle Stösse untersucht werden.

Die in Kapitel \ref{kap:simulation} beschriebene Simulation ist zweidimensional.
Zur korrekten Behandlung von aller Arten von Spin, beispielsweise Sidespin, müsste diese auf eine 3D-Simulation erweitert werden.
Die Suche könnte weiterhin zweidimensional bleiben, würde dadurch aber auch keine Stösse mit Spin vorschlagen.
Dabei könnte evtl. die bereits erwähnte Vereinigung mit einer Vorwärtssuche helfen, welche dem Stoss als
Anfangsbedingung zusätzlich auch Spin geben könnte.

Die Genauigkeit der Klassifikation der Kugelfarben könnte weiter optimiert werden, um bessere Resultate zu erzielen.
Dies ist bei den über den gesamten Tisch unterschiedlichen Lichtverhältnissen und projizierten Augmentationen herausfordernd.

Der Einsatz einer Hochgeschwindigkeitskamera könnte nachgelagerte Analysen eines Spiels,
dessen Aufzeichnung und spätere Wiedergabe erlauben.
Dies würde vermutlich weitere Optimierungen in der Detektion erfordern.

Eine Ausweitung auf andere Arten des Billards, wie etwa Poolbillard, sind durch den Aufbau des Systems möglich, da
Snooker-spezifische Komponenten von Allgemeingültigen getrennt wurden.
Es müsste eine Detektion und Klassifikation für Poolbillard-Kugeln und einige Spielregeln implementiert werden.

Des Weiteren könnte es für den Spieler von Nutzen sein, wenn das System die Position und Orientierung des Billardqueues
erkennen könnte.
Dank dieser Informationen könnte dem Spieler angezeigt werden, wie der Spielball bei einem Stoss in diese Richtung
an den Banden reflektieren würde, um so Erfahrungen mit Stössen über die Banden zu sammeln.
Es gibt bereits verwandte Arbeiten, wo dies umgesetzt ist, siehe Kapitel \ref{kap:verwandte_arbeiten}.
Ausserdem wäre es möglich, verschieden starke Stösse aufgrund der aktuellen Position und Orientierung des Queues zu simulieren
und dem Spieler den besten Stoss animiert anzuzeigen.

Für ein erweitertes Spielerlebnis sind weitere Spielmodi denkbar, beispielsweise die Erkennung und Behandlung von
Regelverstössen, dem Wechsel zwischen zwei Spielern und dem Zählen der Punktzahl jedes Spielers.
In einem Trainingsmodus könnte dem Spieler ein Stoss angezeigt, dessen Ausführung beobachtet und dem Spieler
Rückmeldung gegeben werden.
Über die Darstellung würde es sich anbieten, verschiedene Ereignisse mit speziellen Animationen oder über Ton hervorzuheben,
beispielsweise wenn ein Spieler eine Kugel erfolgreich einlocht, oder wenn der Spielball ebenfalls versenkt wird.

