\newpage
\section{Detektion von Snooker-Kugeln}\label{kap:detektion}
Aufgrund eines Bildes des Billardtisches soll der Spielstand mit der Position aller Kugeln bestimmt werden.
Der in dieser Arbeit verwendete Detektionsalgorithmus stammte von der vorherigen Projektarbeit\cite{project2:snooker_detection}
und funktioniert für Snooker-Kugeln.
In der Vorarbeit wurde die Detektion dafür vorgesehen, per Knopfdruck durchgeführt zu werden.
Mit dieser Arbeit wurde darauf aufbauend eine Live-Detektion implementiert, welche kontinuierlich Bilder vom Tisch macht,
den Spielstand detektiert und dem Benutzer über den Projektor anzeigt.
Dadurch entsteht ein Feedback-Loop, weil die dargestellten Augmentationen im Bild, das Verarbeitet wird, ersichtlich sind.
Ein Eingabebild für die Detektion ist in Abbildung \ref{fig:detection_feedback_loop} dargestellt und zeigt diesen Feedback-Loop.

\begin{figure}[h!]
    \begin{center}
        \includegraphics[width=0.8\linewidth]{../common/03_billiard_ai/resources/detection_feedback_loop.png}
    \end{center}
    \caption{Feedback-Loop in der Detektion}
    \label{fig:detection_feedback_loop}
\end{figure}

Um diesem Feedback-Loop entgegenzuwirken wurden die Parameter der Detektion angepasst.

Des weiteren wurde die Performance der Detektion erhöht, um die Live-Detektion aufzuwerten.
In der bisherigen Detektion wurden drei separate Circle Hough transform\cite{wiki:circle_hough} angewendet, eine für
jede der drei Gruppen von Kugeln, in die das Bild mittels Segmentation aufgeteilt wurde\cite{project2:snooker_detection}.
Zwei dieser drei konnten ohne einen grossen Verlust in der Qualität der Detektion zusammengefasst werden.

% TODO: beschreiben
