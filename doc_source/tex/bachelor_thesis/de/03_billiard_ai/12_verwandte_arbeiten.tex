\section{Verwandte Arbeiten}
Im Umfeld dieser Arbeit existieren bereits einige Lösungen. Es gibt diverse Ansätze, welche
unter anderem auch Roboter miteinbeziehen\cite{qucosa:ein_billardroboter:1} oder sich ganz einer AI eines
optimalen Stosses ohne direkten Realitätsbezug widmen\cite{inproceedings:billiard_ai:1}. Diese Arbeiten werden nicht
direkt als Vergleich hinzugezogen, es wird sich eher auf praktische Lösungen mit ähnlichem Funktionalitätsumfang
konzentriert.

\subsection{Automatic pool stick vs. strangers}
Auf dem Youtube-Kanal \glqq Stuff Made Here\grqq{} werden regelmässig beeindruckende Projekte veröffentlicht, so befasst
sich eine Arbeit ebenfalls mit der Thematik des Billardspiels, wo es um das Bauen eines Billardqueues geht, der einen
über eine Kamera detektierten Stoss ausführen kann\cite{stuffmadehere:automaticpoolstick}. Die Herangehensweise ist
sehr ähnlich, so werden für die Detektion der Kugeln ebenfalls Aruco-Marker eingesetzt und die Suche nach einem optimalen
Stoss geschieht über eine Diskretisierung des Suchraums, in der von jedem Loch aus alle Möglichkeiten berechnet und
bewertet werden. Zusätzlich wurde, wie bereits erwähnt, ein automatischer Queue gebaut, der den Stoss mit dem korrekten
vertikalen Versatz und der passenden Geschwindigkeit ausführen kann. Dieser Queue lässt sich auch über das Internet
steuern, sodass ein Spieler teilnehmen kann, der gar nicht physisch vor Ort ist.

Die vorliegende Arbeit beinhaltet hingegen nicht nur die Anzeige des Suchresultats wie bei \glqq Stuff Made Here\grqq{},
sondern auch deren Simulation. Es werden die Wege aller Kugeln bis zum
Stillstand oder Einlochen auf Basis der physikalischen Eigenschaften des Tisches sowie der Kugeln korrekt animiert.
Dadurch ist die Konsequenz des Stosses bekannt, was wiederum die Möglichkeit der Berücksichtigung
mehrerer aufeinanderfolgender Stösse eröffnet hat.
Diese Ergänzung wird als mögliche weitere Arbeit bei \glqq Stuff Made Here\grqq{} am Ende des Videos erwähnt.

\subsection{Pool Live AR}
Dieses Projekt befasst sich nicht mit der Suche und Visualisierung eines optimalen Stosses, sondern mit der
Detektion des Queues und der Anzeige des resultierenden Weges der weissen Kugel\cite{poollivear}. Zudem wurden einige
Effekte/Animationen eingebaut, die das Spielerlebnis verbessern sollen, z.B. wenn die weisse Kugel angestossen wird.
Das System kann den Weg der weissen Kugel mit einem konstanten Fehlerwinkel und einer bestimmten Anzahl an
Bandenreflektionen anzeigen.
Kollisionen mit anderen Kugeln wie auch deren Konsequenzen für den Spielstand werden nicht erkannt.
Das System hat demnach vermutlich keine Simulation, sondern detektiert die 2D-Orientierung des Queues und reflektiert die
daraus enstehende Linie des Spielballs an den Banden.

\subsection{Innovationsgrad}
Es gibt bereits viele Arbeiten die sich thematisch mit derselben Problematik befassen, sei dies theoretischer oder
praktischer Natur.
Als innovativ kann die animierte Darstellung der Konsequenz eines Stosses wie auch die Suche über
mehrere Spielstände genannt werden.
Ausserdem wurde mit dem Infinity-Modus eine interaktive Möglichkeit geschaffen, um den Spieler mit Stossvorschlägen zu unterstützen.
Dem Spieler wird die benötigte Stärke eines Stosses angezeigt, wodurch dieser lernt, dass viele Stösse im Billard nicht viel
Kraft benötigen, um das Ziel zu erreichen.
Die Suche über mehrere Spielstände verdeutlicht dem Spieler, dass in Snooker die Platzierung des Spielballs essentiell ist
und fördert dessen Weitsicht.

Des Weiteren bietet die Interpretation der Kugeln in Modellkoordinaten Spielraum,
der weitere Innovationen wie z.B. eine Replay-Funktionalität oder andere Verarbeitungen des Spielstandes zulassen.

Physikalische Eigenschaften wie Topspin und dass die Billardkugeln zunächst gleiten und nach einer gewissen Zeit rollen
wurde in der Simulation berücksichtigt.
Durch die Berücksichtigung von Sidespin oder Verbesserungen in der Simulation des Rollverhalten könnte sich das
Projekt noch weiter von der Konkurrenz abheben.
