\subsection{Tiefensuche}
Um dem Spieler einen möglichst optimalen Stoss vorzuschlagen, sollte nicht nur der aktuelle Spielstand,
sondern auch die zukünftigen Spielstände und deren mögliche Stösse berücksichtigt werden.
Ein Stoss ist nur gut, wenn er für die Darauffolgenden eine ebenso gute, wenn nicht bessere, Ausgangslage bietet.
Ein guter Stoss in der aktuellen Situation kann evtl. zu schwierigeren Stössen führen, weil der Spielball schlecht platziert wurde.
Daher wurde beim Lösungsdesign des Suchalgorithmus darauf geachtet, dass eine Suche über mehrere Stösse ermöglicht wird.
Es handelt sich immer noch um eine Graphensuche, deren Suchraum jedoch erweitert wird.
Ein korrespondierender Suchbaum wird in Abbildung \ref{fig:suchbaum_tiefensuche} veranschaulicht.
Die orangenen Knoten bilden die Root-Knoten, die Repräsentation der Ziellöcher. Die grünen Knoten sind
Expansionsknoten. Diese stehen z.B. für einen Weg über eine Kugel oder eine Bande. Die beiden Knoten sind bereits aus
Kapitel \ref{sec:kandidatensuche} bekannt. Sobald die weisse Kugel expandiert wird, ist eine Kandidatensuche beendet und
es startet eine Simulation. Die Simulation wird beim Durchführen des Suchalgorithmus ebenfalls als Knoten in diesem
Suchbaum angesehen, hier blau eingefärbt. Nach der Berechnung eines Simulationsknotens kann die Suche bei genügender
Tiefe beendet oder wie in Ebene 3 fortgeführt werden. Das Resultat des Simulationsknotens bildet den neuen Spielstand
und es werden neue Root-Knoten darauf basierend expandiert. Die Suche kann so zukünftige Stösse in die Bewertung
des aktuellen Stosses einbeziehen.
Das Resultat der Tiefensuche sind nicht mehr nur einzelne Stösse, sondern Abfolgen von Stössen, welche nacheinander
ausgeführt werden könnten.

\begin{figure}[h!]
    \begin{center}
        \includegraphics[width=0.7\linewidth]{../common/03_billiard_ai/resources/38_tiefensuche_suchbaum.png}
    \end{center}
    \caption{Suchbaum der Tiefensuche}
    \label{fig:suchbaum_tiefensuche}
\end{figure}

\subsubsection{Regeln für Snooker}\label{kap:tiefensuche:regeln_fuer_snooker}
Wird über mehrere Stösse gesucht, sind gewisse Regeln des Spiels einzuhalten.
Nachfolgend werden diejenigen aufgeführt, welche für diese Arbeit relevant sind.
Bei Snooker werden abwechslungsweise rote und farbige Kugeln versenkt, solange noch rote Kugeln auf dem Tisch sind.
Die Kugeln ergeben unterschiedliche Punkte: Rot(1), Gelb(2), Grün(3), Braun(4), Blau(5), Pink(6), Schwarz(7) \cite{stoppball:spielregel:snooker}.\\
Gibt es noch eine oder mehrere rote Kugeln zu spielen, wird die versenkte farbige Kugel, ausser die Rote, wieder
auf ihrem Spot (Aufsetzmarke) platziert.
Ein Spot entspricht der Position einer Kugel in der Ausgangsstellung wie in Abbildung \ref{fig:snooker_ausgangslage} dargestellt.
Die Wertung der Spots entspricht der Wertung der Kugeln, welche bei der Ausgangsstellung platziert wurden.

Ist der Spot nicht frei, so wird sie auf den nächsthöherwertigen Spot gelegt.
Wenn alle Spots belegt sind, wird die Kugel so nah wie möglich an ihrem ursprünglichen Spot in Richtung der Kopfbande platziert.
Die Kopfbande ist diejenige, welcher der schwarzen Kugel bei der Ausgangsstellung am nächsten ist \cite{stoppball:spielregel:snooker}.

\begin{figure}[h!]
    \begin{center}
        \includegraphics[width=0.4\linewidth]{../common/03_billiard_ai/resources/39_snooker_ausgangslage.png}
    \end{center}
    \caption{Ausgangslage bei Snooker \cite{stoppball:spielregel:snooker}:
    Die Punktzahlen stehen neben den Kugeln.
    Alle nicht roten Kugeln wurden auf ihrem Spot platziert.)
    }
    \label{fig:snooker_ausgangslage}
\end{figure}

Es wurde ein Algorithmus implementiert, welcher eine möglichst nahe Platzierung der Kugel an den Spot zum Ziel hat.
Dazu wird die Umgebung des Spots nach einem freien Platz für die Kugel durchsucht.
Die genaue Funktionsweise des Algorithmus ist in Kapitel \ref{anhang:spot_platzierung} beschrieben.
