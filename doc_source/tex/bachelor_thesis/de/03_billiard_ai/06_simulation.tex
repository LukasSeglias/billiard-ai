\subsection{Simulation}
Sobald ein möglicher Lösungskandidat anhand der in Abschnitt \ref{sec:kandidatensuche} beschriebenen Suche gefunden
und dessen Initialgeschwindigkeit nach Abschnitt \ref{sec:initialgeschwindigkeit} berechnet wurde,
wird eine Simulation durchgeführt, um die Lösung definitiv zu bestätigen.
Durch das Anwenden verschiedener Anfangsgeschwindigkeiten der weissen Kugel können in diesem Schritt mehrere Situationen evaluiert werden.

Die Simulation wird durch die Definition eines physikalischen Systems wie in Kapitel \ref{kap:physikalisches_system} durchgeführt.
Hierbei gelten die Zuordnungen wie sie nachfolgend beschrieben werden.\\
\textbf{Ereignisse}
\begin{description}
    \item[Energy-Input-Node] Wird modelliert über die Eingabe der Energie der weissen Kugel. Ein spezifischer Node zur
    Modellierung wird nicht implementiert, es wird der Energy-Transfer-Node verwendet, wobei nur der Output-Wert relevant ist.
    \item[Energy-Transfer-Node] Tritt bei der Kollision zwischen zweier Kugeln oder einer Kugel mit der Bande auf.
    \item[No-Energy-Node] Tritt auf, wenn eine Kugel vom dynamischen in den statischen Zustand wechselt (ausrollt). In jedem
    Layer, wo eine Kugel statisch ist, wird sie durch diesen Node modelliert.
    \item[Out-of-System-Node] Sobald eine Kugel mit dem Zielkreis kollidiert, tritt dieses Ereignis auf. Dem System wird die
    Energie entzogen und die Kugel ist nicht mehr verfügbar.
\end{description}

\textbf{Kantenfunktion}
Die Kantenfunktion zwischen den Übergängen innerhalb des Layers bildet der Reibungsverlust der
Kugel über eine bestimmte Zeit oder einen bestimmten Ort.

\textbf{Dynamische/Statische Objekte}
Im Billiard gibt es nur die Kugeln als statische und/oder dynamische Objekte.

\textbf{Konstante Objekte}
Die konstanten Objekte bilden die Banden wie auch die Ziele.
\\

Es wird ungefähr der Pseudoalgorithmus wie in \ref{alg:physikalisches_system} angewendet, optimal auf das Problem \glqq{} Billiard\grqq{}
abgestimmt. Es folgen die physikalischen Berechnungen zur Durchführung der Simulation.

\subsubsection{Reibungsverlust über Strecke}
TODO: T-5 Reibungsverslut über eine Strecke

\subsubsection{Ereignis Out-Of-Energy}
TODO: T-5 Auftrittszeitpunkt Out-Of-Energy über Reibungsverlust

\subsubsection{Ereignis Energy-Transfer über Kugelkollision}
TODO: T-5 Kollision mit Kugeln (statisch/dynamisch) berechnen

\subsubsection{Ereignis Energy-Transfer über Bandenkollision}
TODO: T-5 Kollision mit Banden berechnen.

\subsubsection{Ereignis Out-Of-System}
TODO: T-5 Kollision mit Zielkreis berechnen.