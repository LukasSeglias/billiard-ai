\subsection{Simulation}
Sobald ein möglicher Lösungskandidat anhand der in Abschnitt \ref{sec:kandidatensuche} beschriebenen Suche gefunden
und dessen Initialgeschwindigkeit nach Abschnitt \ref{sec:initialgeschwindigkeit} berechnet wurde,
wird eine Simulation durchgeführt, um die Lösung definitiv zu bestätigen.
Durch das Anwenden verschiedener Anfangsgeschwindigkeiten der weissen Kugel können in diesem Schritt mehrere Situationen evaluiert werden.

Die Simulation wird durch die Definition eines physikalischen Systems wie in Kapitel \ref{kap:physikalisches_system} durchgeführt.
Hierbei gelten die Zuordnungen wie sie nachfolgend beschrieben werden.\\
\textbf{Ereignisse}
\begin{description}
    \item[Energy-Input-Node] Wird modelliert über die Eingabe der Energie der weissen Kugel. Ein spezifischer Node zur
    Modellierung wird nicht implementiert, es wird der Energy-Transfer-Node verwendet, wobei nur der Output-Wert relevant ist.
    \item[Energy-Transfer-Node] Tritt bei der Kollision zwischen zweier Kugeln oder einer Kugel mit der Bande auf.
    \item[No-Energy-Node] Tritt auf, wenn eine Kugel vom dynamischen in den statischen Zustand wechselt (ausrollt). In jedem
    Layer, wo eine Kugel statisch ist, wird sie durch diesen Node modelliert.
    \item[Out-of-System-Node] Sobald eine Kugel mit dem Zielkreis kollidiert, tritt dieses Ereignis auf. Dem System wird die
    Energie entzogen und die Kugel ist nicht mehr verfügbar.
\end{description}

\textbf{Kantenfunktion}
Die Kantenfunktion zwischen den Übergängen innerhalb des Layers bildet der Reibungsverlust der
Kugel über eine bestimmte Zeit oder einen bestimmten Ort.

\textbf{Dynamische/Statische Objekte}
Im Billiard gibt es nur die Kugeln als statische und/oder dynamische Objekte.

\textbf{Konstante Objekte}
Die konstanten Objekte bilden die Banden wie auch die Ziele.

Es wird ungefähr der Pseudoalgorithmus wie in \ref{alg:physikalisches_system} angewendet, optimal auf das Problem \glqq{} Billiard\grqq{}
abgestimmt. Es folgen die physikalischen Berechnungen zur Durchführung der Simulation.


\subsubsection{Reibungsverlust über die Zeit}
Die Geschwindigkeit einer Kugel wird durch die Reibung über die Zeit reduziert. Dazu wird die Formel der gleichförmig
beschleunigten Bewegung verwendet, wobei $\vec{v_0}$, $\vec{a}$ sowie $t$ gegeben sind.
\begin{align}
    \vec{v} = \vec{a} \cdot t + \vec{v_0}
\end{align}


\subsubsection{Elastischer Stoss zweier Kugeln}
Die Kollision zweier Kugeln wird im folgenden als zweidimensionaler elastischer Stoss angesehen. Dabei ist es wichtig,
zwei Komponenten beim Zusammenprall zu betrachten. Dies ist einerseits das Liniensegment $s_z$ zwischen den Mittelpunkten
der Kugeln sowie die orthogonal dazu stehende Gerade $s_t$. Die Kugel, welche Energie mitbringt, übergibt der anderen
Kugel Energie in Richtung von $s_z$. Die übrig gebliebene Energie zeigt in Richtung von $s_t$ \cite{wiki.elastischer_stoss_physik:1}.
In Abbildung \ref{fig:Elastischer Stoss zweier Kugeln} ist ein solcher Stoss dargestellt.
Kugel eins bringt dabei die Geschwindigkeit $\vec{v_1}$ und Kugel zwei die Geschwindigkeit $\vec{v_2}$ mit. Ein Anteil
der Geschwindigkeit von Kugel eins wird der Kugel zwei in Richtung von $s_z$ übergeben. Dasselbe gilt für die
Kugel zwei. Sie übergibt einen Teil ihrer Geschwindigkeit an Kugel eins ebenfalls in Richtung von $s_z$. Die verbleibende
Energie der Kugeln zeigt jeweils in Richtung von $s_t$.
\begin{figure}[h!]
    \begin{center}
        \includegraphics[width=0.4\linewidth]{../common/03_billiard_ai/resources/23_elastischer_stoss.png}
    \end{center}
    \caption{Elastischer Stoss zweier Kugeln}
    \label{fig:Elastischer Stoss zweier Kugeln}
\end{figure}

Um nun die neuen Geschwindigkeiten $\vec{u_1}$ und $\vec{u_2}$ zu berechnen, müssen die initialen Geschwindigkeiten
$\vec{v_1}$ sowie $\vec{v_2}$ auf die Komponenten in Richtung von $s_z$ und $s_t$ aufgeteilt werden.
\begin{align}
    \vec{z} = \vec{Z_2} - \vec{Z_1}\\
    \hat{z} = \frac{\vec{z}}{\norm{\vec{z}}}
\end{align}
Von Interesse ist dabei nur $\hat{z}$. Mittels diesem Vektor kann die Komponente in Richtung von $s_z$ beider Geschwindigkeiten
ausgerechnet werden.
\begin{align}
    \vec{v_{1,z}} = (\vec{v_1} \cdot \hat{z}) \cdot \hat{z}\\
    \vec{v_{2,z}} = (\vec{v_2} \cdot \hat{z}) \cdot \hat{z}
\end{align}
Mittels dieses neuen Vektors in $s_z$ Richtung kann nun der Vektor in $s_t$ Richtung berechnet werden.
\begin{align}
    \vec{v} = \vec{v_t} + \vec{v_z}\\
    \vec{v_t} = \vec{v} - \vec{v_z}
\end{align}
Daraus folgt für die jeweiligen Kugeln:
\begin{align}
    \vec{v_{1,t}} = \vec{v_1} - \vec{v_{1,z}}\\
    \vec{v_{2,t}} = \vec{v_2} - \vec{v_{2,z}}
\end{align}
Die neuen Geschwindigkeitsvektoren setzen sich nun wie beschrieben aus den jeweiligen Komponenten zusammen.
Das Resultat lautet nun wie folgt:
\begin{align}
    \vec{u_1} = \vec{v_{2,z}} + \vec{v_{1,t}}\\
    \vec{u_2} = \vec{v_{1,z}} + \vec{v_{2,t}}
\end{align}
Somit können bei gegebener Kollision zweier Kugeln die Geschwindigkeiten nach der Kollision bestimmt werden.


\subsubsection{Bandenreflektion}
Sofern eine Kugel an eine Bande stösst, so wird diese abgelenkt. In dem hier beschriebenen Modell wird der Drall\cite{wiki.spin:1},
welcher die Bahn einer Kugel nach Kollision mit der Bande ablenken würde, ignoriert.
Das bedeutet, es wird davon ausgegangen, dass der Ausfallswinkel nach der Bandenreflektion gleich dem Eifallswinkel sei.
Dazu kann die folgende Formel\cite{paulbourke.reflected_ray:1} verwendet werden, wobei $I$ der einfallende
und $R$ der ausgehende Weg der Kugel und $N$ der Normalenvektor der Bande sind:
\begin{align}
    R = I - 2 \cdot N \cdot (I \cdot N)
\end{align}


\subsubsection{Kollisionsprüfung}
Während der Simulation ist es notwendig zu prüfen, welche Kugeln mit welchen kollidieren könnten.
Um zu wissen, ob eine Kugel eine Strecke zurücklegen kann, ohne mit einer anderen Kugel zu kollidieren,
muss für jede andere Kugel geprüft werden, ob diese im Weg liegt. Daher sollte dieser Test effizient sein.
Für den Test wird die zurückzulegende Strecke als Liniensegment zwischen Punkt $A$ und Punkt $B$ und die Position
einer zu testenden Kugel $C$ als Punkt verstanden. Anschliessend wird der Abstand zwischen dem Punkt $C$ und dem
Liniensegment $AB$ geprüft, ob dieser kleiner als der Kugeldurchmesser ist. Sofern dies der Fall ist, liegt die Kugel an
der Position $C$ im Weg und es würde eine Kollision stattfinden, sofern die Ausgangskugel die Strecke $AB$ rollt.
Dies wird für jede Kugel geprüft.


\subsubsection{Ereignis Out-Of-Energy}
TODO: T-5 Auftrittszeitpunkt Out-Of-Energy über Reibungsverlust

\subsubsection{Ereignis Energy-Transfer über Kugelkollision}
TODO: T-5 Kollision mit Kugeln (statisch/dynamisch) berechnen

\subsubsection{Ereignis Energy-Transfer über Bandenkollision}
Dieses Event beschreibt die Kollision mit der Bande. Es soll der Zeitpunkt $t$ der Kollision mit der Bande festgestellt werden.
Der Algorithmus funktioniert so, dass zuerst geprüft wird, ob eine Kollision stattfinden kann.
Dies erfolgt über einen Schnittpunkt-Test zwischen einer Linie und einem Liniensegment.
Eine Bande kann als Liniensegment zwischen dem Startpunkt $R_1$ und $R_2$ betrachtet werden. Diese Punkte müssen demnach bekannt sein.
Weiterhin ist der Geschwindigkeitsvektor $\vec{v}$ und die Position der Kugel $C$ bekannt.
Aufgrund dieser Informationen kann eine Linie definiert werden.

Die Punkte $R_1$ und $R_2$ werden um den Kugelradius zur Tischmitte verschoben,
damit dem Kugelradius Rechnung getragen wird und dieser nicht weiter betrachtet werden muss.
Daraus ergeben sich $R_1'$ und $R_{2'}$\footnote{Für Herleitung, siehe Anhang \ref{anhang:herleitung:event:collisionWithRail}}.

Erster Schritt - Prüfe, ob Kollision möglich:\\
Dazu kann die in Anhang \ref{anhang:herleitung:event:collisionWithRail} erarbeitete Formel mit einigen Variablensubstitutionen verwendet werden:
\begin{align}
    P = \vec{R_1'}\\
    D = \vec{\Delta R'}\\
    Q = \vec{C}\\
    V = \vec{v}\\
    \lambda_1 = \frac{V_x \cdot P_y - V_y \cdot P_x + Q_x \cdot V_y - Q_y \cdot V_x}{D_x \cdot V_y - \cdot D_y \cdot V_x}\\
    \lambda_2 = \frac{P_x \cdot D_y - P_y \cdot D_x - Q_x \cdot D_y + Q_y \cdot D_x}{V_x \cdot D_y - V_y \cdot D_x}
\end{align}

Betrachte den Nenner eines der $\lambda$. Ergibt dieser $0$, so gibt es keinen Schnittpunkt
und eine weitere Betrachtung erübrigt sich. Ansonsten muss für $\lambda$ folgendes gelten:
\begin{align}
    0 \leq \lambda_1 \leq 1\\
    0 \leq \lambda_2
\end{align}
Das bedeutet, dass der Schnittpunkt mit der Bande in positiver Richtung des Geschwindigkeitsvektors der Kugel sowie an
der effektiven Bande, also am Segment zwischen den Punkten $R_1'$ und $R_2'$, sein muss.
Ist dies gegeben und beide $\lambda$ erfüllen die Bedingungen, dann kann der Zeitpunkt der Kollision berechnet werden.

Zweiter Schritt - Ort der Kollision bestimmen:\\
Betrachte dazu eine der Gleichungen mit zugehörigem $\lambda$:
\begin{align}
    \vec{s(\lambda_1)} = \vec{R_1'} + \lambda_1 \cdot \vec{\Delta R'}\\
    \vec{s(\lambda_2)} = \vec{C} + \lambda_2 \cdot \vec{v}
\end{align}

Dritter Schritt - Zeitpunkt der Kollision bestimmen:\\
Dies kann über die folgenden Gleichungen gelöst werden:
\begin{align}
    t_1 = \frac{-v_x + \sqrt{v_x^2 - 2 \cdot a_x \cdot \Delta s_x}}{a_x}\\
    t_2 = \frac{-v_x - \sqrt{v_x^2 - 2 \cdot a_x \cdot \Delta s_x}}{a_x}
\end{align}
Bevor die Lösung zu $t_1$ und $t_2$ berechnet wird, muss jedoch die Radikale auf folgende Eigenschaft geprüft werden:
\begin{align}
    0 \leq v_x^2 - 2 \cdot a_x \cdot \Delta s_x
\end{align}
Nur in diesem Fall gibt es eine Lösung und somit an dieser Position eine Kollision. Ansonsten hat die Kugel vorher schon
ihre Gesamtenergie verloren und steht still. Damit ist der Zeitpunkt der Kollision berechnet.

\subsubsection{Ereignis Out-Of-System}
TODO: T-5 Kollision mit Zielkreis berechnen.
