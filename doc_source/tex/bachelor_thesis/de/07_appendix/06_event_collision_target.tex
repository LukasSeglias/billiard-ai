\section{Herleitung Ereignis - Kollision mit Ziel}\label{anhang:herleitung:event:targetCollision}
Um den Kollisionszeitpunkt zu bestimmen, wird in einem ersten Schritt der Schnittpunkt des Weges der Kugel mit
dem Zielkreis berechnet. Dazu wird die Kreisgleichung wie in \ref{eq:target_collision_event:00} verwendet.
\begin{align}
    \norm{X - C}^2 &= r^2\\
    (x - C_x)^2 + (y - C_y)^2 &= r^2\label{eq:target_collision_event:00}
\end{align}

Hierbei gelten die Definitionen aus \ref{eq:target_collision_event:99}.
\begin{align}
    \hat{d} = \frac{\vec{d}}{\norm{\vec{d}}}\label{eq:target_collision_event:99}
\end{align}

Die Parameterform der Geradengleichung ist gegeben in \ref{eq:target_collision_event:01}.
\begin{align}
    \vec{d} &= \vec{p} + l \cdot \hat{v}\\
    \begin{pmatrix}d_{x}\\d_{y}\end{pmatrix} &= \begin{pmatrix}p_{x}\\p_{y}\end{pmatrix} + l \cdot \begin{pmatrix}v_{x}\\v_{y}\end{pmatrix}\label{eq:target_collision_event:01}
\end{align}

Durch Einsetzen der Gleichung \ref{eq:target_collision_event:01} in \ref{eq:target_collision_event:00} folgt \ref{eq:target_collision_event:02}.
\begin{align}
    (p_x + l \cdot v_x - C_x)^2 + (p_y + l \cdot v_y - C_y)^2 = r^2\label{eq:target_collision_event:02}
\end{align}

Im nächsten Schritt wird die Gleichung \ref{eq:target_collision_event:02} nach t umgeformt.
\begin{align}
    (p_x + l \cdot v_x - C_x)^2 + (p_y + l \cdot v_y - C_y)^2 &= r^2\\
    \norm{\vec{p} + l \cdot \hat{v} - \vec{C}}^2 &= r^2\\
    \norm{l \cdot \hat{v} + \vec{p} - \vec{C}}^2 &= r^2\\
    \norm{l \cdot \hat{v} + (\vec{p} - \vec{c})}^2 &= r^2\\
    (l \cdot \hat{v} + (\vec{p} - \vec{C})) \cdot (l \cdot \hat{v} + (\vec{p} - \vec{C})) &= r^2\\
    l^2 \cdot (\hat{v} \cdot \hat{v}) + 2l \cdot (\hat{v}(\vec{p} - \vec{C})) + (\vec{p} - \vec{C}) \cdot (\vec{p} - \vec{C}) &= r^2\\
    l^2 \cdot (\hat{v} \cdot \hat{v}) + 2l \cdot (\hat{v}(\vec{p} - \vec{C})) + (\vec{p} - \vec{C}) \cdot (\vec{p} - \vec{C}) - r^2 &= 0\label{eq:target_collision_event:03}
\end{align}

Die Gleichung \ref{eq:target_collision_event:02} ist von einer quadratischen Form und kann mithilfe der Mitternachtsformel
gelöst werden.
\begin{align}
    a \cdot l^2 + b \cdot l + c &= 0\\
    a &= \hat{v} \cdot \hat{v} = \norm{\hat{v}}^2 = 1\\
    b &= 2 \cdot \hat{v} \cdot (\vec{p} - \vec{C})\\
    c &= (\vec{p} - \vec{C}) \cdot (\vec{p} - \vec{C}) - r^2
\end{align}

Die Lösungen lauten demnach:
\begin{align}
    l_1 &= \frac{-b + \sqrt{b^2 - 4ac}}{2a}\\
    l_2 &= \frac{-b - \sqrt{b^2 - 4ac}}{2a}
\end{align}
Wobei $b^2 - 4ac$ die Diskriminante ist. Hat diese einen negativen Wert, existiert kein Schnittpunkt,
beim Wert $0$ gibt es einen Schnittpunkt und bei einem Wert grösser als $0$ gibt es zwei Schnittpunkte.

Die Lösung lautet demnach:
\begin{align}
    l_1 &= \frac{-(2 \cdot \hat{v} \cdot (\vec{p} - \vec{C})) + \sqrt{4 \cdot ((\vec{p} - \vec{C}) \cdot (\vec{p} - \vec{C}) - r^2)}}{2}\\
    l_2 &= \frac{-(2 \cdot \hat{v} \cdot (\vec{p} - \vec{C})) - \sqrt{4 \cdot ((\vec{p} - \vec{C}) \cdot (\vec{p} - \vec{C}) - r^2)}}{2}\\
    l &= \max{(\min{(l_1, l_2)}, 0)}
\end{align}

Ein $l$ ist valid, wenn es positiv ist und nicht innerhalb des Ziellochs ($-l_1, +l_2$) liegt.
Dieser Fall wird nicht vorkommen, da die Kugel bei der ersten Kollision aus dem System entfernt wird.
Sollte ein valides $l$ existieren, kann damit die Distanz $d$ zwischen der Kugel und dem Schnittpunkt bestimmt werden.
\begin{align}
    \vec{d(l)} = l \cdot \hat{v}\label{eq:target_collision_event:04}
\end{align}

Anhand der Distanz aus \ref{eq:target_collision_event:04} kann der Zeitpunkt durch die Gleichung \ref{eq:target_collision_event:05}
bestimmt werden, wobei gilt $\vec{s(t)} = \vec{d(l)}$
\begin{align}
    \vec{s(t)} &= \frac{1}{2} \cdot \vec{a} \cdot t^2 + \vec{v} \cdot t\\\label{eq:target_collision_event:05}
    \frac{1}{2} \cdot \vec{a} \cdot t^2 + \vec{v} \cdot t - \vec{d(l)} &= 0
\end{align}

Da es sich um die Berechnung der benötigten Zeit von der Kugel zum Ziel handelt und keine weiteren Faktoren relevant sind,
kann die Berechnung der Zeit entweder auf eine Dimension reduziert werden oder die X- wie auch Y-Dimensionen werden separat
berechnet. Dieselbe Sachlage wurde bereits in Kapitel \ref{anhang:herleitung:event:outOfEnergy} erläutert. Da es sich
hier um eine quadratische Gleichung handelt, wäre hingegen der Mehraufwand der separaten Betrachtung der X- und Y-Dimensionen
höher als die Reduktion auf eine Dimension durch die Betragsnahme der Vektoren, weswegen der zweitgenannte Weg verfolgt
wird. Demnach ergibt sich die Situation in \ref{eq:target_collision_event:07}.

\begin{align}
    a \cdot t^2 + b \cdot t + c &= 0\\\label{eq:target_collision_event:07}
    a &= \frac{1}{2} \cdot -\norm{\vec{a}}\\
    b &= \norm{\vec{v}}\\
    c &= -\norm{\vec{d(l)}}
\end{align}

Es ergeben sich wiederum zwei Lösungen, wobei nur die Zeit des ersten Schnittpunktes (die Minimalzeit) relevant ist.
\begin{align}
    t_1 &= \frac{-b + \sqrt{b^2 - 4ac}}{2a}\\
    t_2 &= \frac{-b - \sqrt{b^2 - 4ac}}{2a}\\
    t &= \min{(t_1, t_2)}
\end{align}

Sollte die Diskriminante kleiner denn $0$ sein, bedeutet dies, dass die Geschwindigkeit der Kugel nicht ausreichend
ist, um den Zielkreis zu erreichen. Es existiert demnach keine Lösung. Die Lösung lautet demnach:
\begin{align}
    t_1 &= \frac{-\norm{\vec{v}} + \sqrt{\norm{\vec{v}}^2 + 2 \cdot -\norm{\vec{a}} \norm{\vec{d(l)}}}}{-\norm{\vec{a}}}\\
    t_2 &= \frac{-\norm{\vec{v}} - \sqrt{\norm{\vec{v}}^2 + 2 \cdot -\norm{\vec{a}} \norm{\vec{d(l)}}}}{-\norm{\vec{a}}}\\
    t &= \min{(t_1, t_2)}
\end{align}