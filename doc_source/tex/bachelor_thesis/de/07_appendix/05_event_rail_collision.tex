\section{Herleitung Ereignis - Kollision mit Bande}\label{anhang:herleitung:event:collisionWithRail}
Das Ziel ist es, die Position einer Kugel zum Zeitpunkt einer Kollision mit einer Bande zu bestimmen.

Die Bande wird über zwei Punkte beschrieben $R_1$ und $R_2$ (für Rail).

Auf Basis dieser Punkte kann eine Geradengleichung formuliert werden:
\begin{align}
    \vec{\Delta R} &= \vec{R_2} - \vec{R_1}\\
    \vec{s(\lambda_1)} &= \vec{R_1} + \lambda_1 \cdot \vec{\Delta R}
\end{align}

Weiterhin kann aufgrund der Position $C$ und der bekannten Geschwindigkeit $\vec{v}$ der Kugel eine Linie definiert werden:
\begin{align}
    \vec{s(\lambda_2)} = \vec{C} + \lambda_2 \cdot \vec{v}\label{eq:event_rail_collision:line_equation_ball}
\end{align}

Der Kollisionspunkt mit der Bande ist damit der Schnittpunkt dieser Linie und dem Banden-Liniensegment.
Nun kann der Schnittpunkt der beiden Geraden anhand der Formel in Anhang \ref{anhang:herleitung:linie-linie-schnittpunkt} bestimmt werden.
Dabei sind einige Variablensubstitutionen zu tun:
\begin{align}
    P &= \vec{R_1}\\
    D &= \vec{\Delta R}\\
    Q &= \vec{C}\\
    V &= \vec{v}\\
\end{align}
Damit gilt dann:
\begin{align}
    \lambda_1 &= \frac{V_x \cdot P_y - V_y \cdot P_x + Q_x \cdot V_y - Q_y \cdot V_x}{D_x \cdot V_y - \cdot D_y \cdot V_x}\\
    \lambda_2 &= \frac{P_x \cdot D_y - P_y \cdot D_x - Q_x \cdot D_y + Q_y \cdot D_x}{V_x \cdot D_y - V_y \cdot D_x}
\end{align}
\iffalse % Nicht im Dokument enthalten
Nach der Variablensubstitution gilt:
\begin{align}
    \lambda_1 &= \frac{\vec{v}_x \cdot \vec{R_1}_y - \vec{v}_y \cdot \vec{R_1}_x + \vec{C}_x \cdot \vec{v}_y - \vec{C}_y \cdot \vec{v}_x}{\vec{\Delta R}_x \cdot \vec{v}_y - \cdot \vec{\Delta R}_y \cdot \vec{v}_x}\\
    \lambda_2 &= \frac{\vec{R_1}_x \cdot \vec{\Delta R}_y - \vec{R_1}_y \cdot \vec{\Delta R}_x - \vec{C}_x \cdot \vec{\Delta R}_y + \vec{C}_y \cdot \vec{\Delta R}D_x}{\vec{v}_x \cdot \vec{\Delta R_y - \vec{v}_y \cdot \vec{\Delta R}_x}
\end{align}
\fi
Es findet nur dann eine Kollision statt, wenn der Nenner der $\lambda$ ungleich $0$ ist.

Da die Bande nur ein Liniensegment darstellt, muss $\lambda_1$ zwischen $0$ und $1$ liegen.
Ausserdem muss $\lambda_2$ grösser oder gleich $0$ sein, da der Weg der Kugel eine Half-Line darstellt und ansonsten
Schnittpunkte hinter der Kugel möglich wären.

Sofern die beiden $\lambda$ diese Bedingungen einhalten, dann findet eine Kollision statt.

Nun soll der Ort $s(t)$ und die Zeit $t$ der Kollision
festgestellt werden. Da der Kugelradius berücksichtigt werden muss,
werden die beiden Punkte $R_1$ und $R_2$ um den Kugelradius zur Tischmitte verschoben.
Diese verschobenen Punkte seien $R_1'$ und $R_2'$.

Mithilfe dieser Punkte kann eine Linie definiert werden:
\begin{align}
    \vec{\Delta R'} &= \vec{R_2'} - \vec{R_1'}\\
    \vec{s(\lambda_1)} &= \vec{R_1'} + \lambda_1 \cdot \vec{\Delta R'}
\end{align}

Der Schnittpunkt dieser Linie mit der in \ref{eq:event_rail_collision:line_equation_ball} definierten Linie ist
der Kollisionspunkt $s(t)$ der Kugel.
Angewandt auf die in Anhang \ref{anhang:herleitung:linie-linie-schnittpunkt} definierte Formel gelten folgende
Variablensubstitutionen:
\begin{align}
    P &= \vec{R_1'}\\
    D &= \vec{\Delta R'}\\
    Q &= \vec{C}\\
    V &= \vec{v}\\
\end{align}
Damit gilt dann:
\begin{align}
    \lambda_1 &= \frac{V_x \cdot P_y - V_y \cdot P_x + Q_x \cdot V_y - Q_y \cdot V_x}{D_x \cdot V_y - \cdot D_y \cdot V_x}\\
    \lambda_2 &= \frac{P_x \cdot D_y - P_y \cdot D_x - Q_x \cdot D_y + Q_y \cdot D_x}{V_x \cdot D_y - V_y \cdot D_x}
\end{align}

Der Ort der Kollision $s(t)$ kann nun mit $\lambda_1$ berechnet werden.

Der Zeitpunkt $t$ der Kollision wird mit der Formel der gleichmässig beschleunigten Bewegung berechnet.
\begin{align}
    \vec{s(t)} = \frac{1}{2} \cdot \vec{a} \cdot t^2 + \vec{v} \cdot t + \vec{s_0}
\end{align}

Die Beschleunigung $\vec{a}$, Geschwindigkeit $\vec{v}$, Startposition $\vec{s_0}$ zum Zeitpunkt $t_0$
und die Position $\vec{s_1}$ zum Zeitpunkt $t_1$ sind bekannt.

Da die Startposition der Kugel zum Zeitpunkt $t_0$ bekannt ist,
kann die Formel auch mit der zurückzulegenden Strecke $\Delta \vec{s}$ definiert werden:
\begin{align}
    t &= t_1 - t_0\\
    \Delta \vec{s} &= \vec{s_1} - \vec{s_0}\\
    \Delta \vec{s} &= \frac{1}{2} \cdot \vec{a} \cdot t^2 + \vec{v} \cdot t + \vec{s_0} - \vec{s_0}\\
    \Delta \vec{s} &= \frac{1}{2} \cdot \vec{a} \cdot t^2 + \vec{v} \cdot t
\end{align}

Das zweidimensionale Problem kann auf eine Dimension reduziert werden, indem die Beträge der Vektoren verwendet werden,
weil die Vektoren $\Delta \vec{s}$, $\vec{a}$ und $\vec{v}$ kollinear sind:
\begin{align}
    \Delta s &= \norm{\Delta \vec{s}}\\
    a &= -\norm{\vec{a}}\\
    v &= \norm{\vec{v}}\\
    \norm{\Delta \vec{s}} &= \frac{1}{2} \cdot (-\norm{\vec{a}}) \cdot t^2 + \norm{\vec{v}} \cdot t\\
    \Delta s &= \frac{1}{2} \cdot a \cdot t^2 + v \cdot t
\end{align}

Umgeformt zu einer quadratischen Gleichung folgt:
\begin{align}
    0 = \frac{1}{2} \cdot a \cdot t^2 + v \cdot t - \Delta s
\end{align}

Diese Gleichung kann nun mit der allgemeinen Lösungsformel für quadratische Gleichungen nach $t$ gelöst werden \cite{wiki.mitternachtsformel:1}:
\begin{align}
    t_{1,2} = \frac{-b \pm \sqrt{b^2 - 4ac}}{2a}
\end{align}
Diese beiden $t_1$ und $t_2$ sind nicht zu verwechseln mit den gleichnamigen Variablen zuvor, es handelt sich um Lösungen für $t$.

Wobei für die Koeffizienten gilt:
\begin{align}
    a &= \frac{1}{2} \cdot a\\
    b &= v\\
    c &= - \Delta s\\
\end{align}

Die Lösungen lauten daher:
\begin{align}
    \Delta t_1 &= \frac{-v + \sqrt{v^2 + 2 \cdot a \cdot \Delta s}}{a}\\
    \Delta t_2 &= \frac{-v - \sqrt{v^2 + 2 \cdot a \cdot \Delta s}}{a}
\end{align}

Die minimale Zeit repräsentiert den Ereigniszeitpunkt, weswegen nur diese verwendet wird.
\begin{align}
    t = \min(t_1, t_2)
\end{align}

Da es sich bei $t$ nur um die vergangene Zeit von dem Startzeitpunkt $t_0$ bis zum Kollisionszeitpunkt $t_1$ handelt,
muss der Startzeitpunkt wieder aufaddiert werden, sofern der Absolute Zeitpunkt relevant ist:
\begin{align}
    t_1 = t_0 + t
\end{align}
