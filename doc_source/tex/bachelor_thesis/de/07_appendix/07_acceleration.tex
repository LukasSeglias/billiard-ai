\section{Herleitung Beschleunigung}\label{anhang:herleitung:beschleunigung}
Die Beschleunigung $a$ einer Kugel, welche durch die Rollreibung auf dem Tisch entsteht,
kann über die wirkenden Kräfte ausgedrückt werden.
Die Länge des Vektors $\vec{a}$ kann folgendermassen ausgedrückt werden \cite{wiki.rollreibung:1},
wenn die Reibungskraft $F_R$ sowie die Normalkraft $F_N$ bekannt ist:

\begin{align}
    F_R = m \cdot a\\
    a = \frac{F_R}{m}\\
    F_N = m \cdot g\\
    F_R = F_N \cdot c_R\\
    a = g \cdot c_R
\end{align}

Die Richtung des Vektors $\vec{a}$ zeigt in die entgegengesetzte Richtung von $\vec{v}$.
Dazu wird $\vec{v}$ normiert. Damit kann $\vec{a}$ bestimmt werden:
\begin{align}
    \vec{a} = g \cdot c_R \cdot \frac{\vec{v}}{\norm{\vec{v}}}
\end{align}
