\section{Verbesserung Simulationsalgorithmus}\label{anhang:simulation:algorithmus}
Zurzeit hat der Simulationsalgorithmus zwei Probleme. Er kann einerseits nur ein Ereignis zu einem Zeitpunkt bestimmen und
andererseits werden alle Ereignisse ab einem Zeitpunkt neu berechnet. Neu soll demnach die Möglichkeit bestehen, mehrere
Ereignisse zum selben Zeitpunkt zuzulassen und alle Ereignisse vorgängig einmal zu berechnen und entsprechend ihrem Eintreten
abzuarbeiten. Sobald eine Kugel an einem Ereignis beteiligt ist, werden sämtliche gespeicherten Ereignisse für diese Kugel und
ihre Partnerkugel gelöscht. Zudem werden alle Ereignisse bei anderen Kugeln entfernt, die mit an dem eingetretenen Ereignis beteiligten Kugeln
zusammenhängen. Danach wird ein neuer Layer zu diesem Zeitpunkt berechnet. Wenn das System nicht konstant ist, werden
für sämtliche Kugeln die Ereignisse neu berechnet, die keine mehr haben. Dabei werden auch die gelöschten Ereignisse
mit am vorherigen Ereignis unbeteiligten Kugeln ergänzt, sollte es solche geben.

\begin{algorithm}[H]
    \DontPrintSemicolon
    \SetKwFunction{simulate}{simulate}
    \SetKwFunction{events}{events}
    \SetKwFunction{append}{append}
    \SetKwProg{Fn}{Function}{}{}
    struct Event \{\\
    timestamp: datetime\\
    me: string\\
    partner: string\\
    event: [OutOfEnergy, Collision, OutOfSystem, Rolling]\\
    \}\\
    \;
    \Fn{\simulate{start: Layer, constantObjects: list} $\longrightarrow$ System}{
        system $\longleftarrow$ System()\\
        system $\longleftarrow$ appendLayer(system, start)\\
        events: Map[string, Event[]] = Map()\\

        \While{! system.isStatic()}{
            events $\longleftarrow$ events(system.lastLayer(), constantObjects, events)\\
            nextEvents $\longleftarrow$ nextEvents(events)\\
            events $\longleftarrow$ delete(events, nextEvents.second)\\
            events $\longleftarrow$ subtractTime(events, nextEvents.first)\\
            layer $\longleftarrow$ atMoment(nextEvents)\\
            system $\longleftarrow$ appendLayer(system, layer)
        }
        \KwRet system
    }
    \;
    \Fn{\events{layer: Layer, constantObjects: list, events: Map[string, Event[]]} $\longrightarrow$ Map[string, Event[]]}{
        \For{object in layer.dynamicObjects()}{
            \If{object.id not in events.keys}{
                events = append(outOfEnergy(object), events)\\
                events = append(collision(object, layer.dynamicObjects()), events)\\
                events = append(collision(object, layer.staticObjects()), events)\\
                events = append(collision(object, constantObjects), events)\\
                events = append(outOfSystem(object, constantObjects), events)\\
                events = append(rolling(object), events)\\
            }
        }
        \KwRet events
    }
    \;
    \Fn{\append{event: Event, events: Map[string, Event[]]} $\longrightarrow$ Map[string, Event[]]}{
        events $\longleftarrow$ append(event.me, event, events)\\
        \If{event.me != event.partner}{
            events $\longleftarrow$ append(event.partner, event, events)\\
        }
        \KwRet events
    }
    \caption{Algorithmus zum Aufbau eines physikalischen Systems - Part 1}
    \label{alg:physikalisches_system_verbesserung_algorithmus_1}
\end{algorithm}

Algorithmus \ref{alg:physikalisches_system_verbesserung_algorithmus_1} beinhaltet die Funktionen für den Simulationsvorgang
an sich, die Berechnung der nächsten Ereignisse und das Hinzufügen neuer Ereignisse. Der Simulationsvorgang sieht im
Wesentlichen so aus, dass solange simuliert wird, wie das System Energie führt. Es werden nacheinander die Ereignisse berechnet,
die nächst eintretenden Ereignisse bestimmt, die gespeicherten Ereignisse darauf basierend bereinigt und bei den übrig
gebliebenen Ereignissen die Zeit korrigiert. Abschliessend kann der Zustand zu diesem Zeitpunkt bestimmt und
dem System hinzugefügt werden. Die Berechnung der Ereignisse wird nur auf dynamische Kugeln angewendet, die keine Ereignisse haben.
Dies ist bei der ersten Ausführung der Fall oder wenn eine Kugel vorgängig an einem Ereignis beteiligt gewesen ist.
Die Funktion zum Hinzufügen des Ereignisses ruft eine Subfunktion auf, die das Ereignis nur hinzufügt, wenn es nicht bereits vorhanden ist.
Zudem wird das Ereignis ebenfalls für das Partnerobjekt erfasst, dies ist insbesondere bei statischen Kugeln notwendig.

\newpage
Der Algorithmus \ref{alg:physikalisches_system_verbesserung_algorithmus_2} beinhaltet die Funktionen zum Löschen von
Ereignissen, zur Bestimmung der nächsten Ereignisse und zur Korrektur der Zeit bei den bekannten Ereignissen.
Die Löschfunktion nimmt einerseits die aktuellen Ereignisse pro Kugel und die nachfolgend eintretenden Ereignisse
entgegen. Als Rückgabe erfolgen wiederum die Ereignisse pro Kugel.
Es werden nacheinander pro eintretendes Ereignis je Kugel alle weiteren vorgesehenen Ereignisse aus der Datenstruktur gelöscht,
was zur Folge hat, dass sie im nächsten Durchgang neu berechnet werden.
Anschliessend werden alle Ereignisse bei anderen Kugeln entfernt, an denen die bereits genannten beteiligt sind.
Bei der Berechnung der nächsten Ereignisse werden die Ereignisse pro Kugel benötigt.
Als Resultat erfolgt der Zeitpunkt ausgedrückt als Delta (von vorherigem Ereignis zum nächsten Ereignis)
und eine Liste von Ereignissen, da mehrere zum selben Zeitpunkt auftreten können.
Es werden alle Ereignisse geprüft und entweder übernommen, wenn der Eintrittszeitpunkt vor dem
aktuell gespeicherten liegt oder der Liste hinzugefügt, wenn dieser dem aktuell gespeicherten entspricht.
Eventuell könnte beim Einfügen der Ereignisse bereits eine Sortierung nach Zeitpunkt stattfinden, somit müssten nur
alle ersten Ereignisse pro Kugel geprüft werden. Bei der Korrektur der Zeit erfolgt die Eingabe
aller Ereignisse und der Deltazeit zum nächsten Ereignis. Die Funktion zieht diese Deltazeit von den
Zeitpunkten ab und gibt als Resultat die korrigierte Struktur zurück.
\begin{algorithm}[H]
    \DontPrintSemicolon
    \SetKwFunction{delete}{delete}
    \SetKwFunction{nextEvents}{nextEvents}
    \SetKwFunction{subtractTime}{subtractTime}
    \SetKwProg{Fn}{Function}{}{}
    \Fn{\delete{events: Map[string, Event[]], nextEvents: Event[]} $\longrightarrow$ Map[string, Event[]]}{
        \For{event in nextEvents}{
            events $\longleftarrow$ drop(event.me, events)\\
            events $\longleftarrow$ drop(event.partner, events)\\
            \For{pair in events}{
                \For{anEvent in pair.second}{
                    \If{anEvent.partner == event.me or anEvent.partner == event.partner}{
                        cleaned $\longleftarrow$ drop(anEvent.partner, pair.second)\\
                        events[pair.first] $\longleftarrow$ cleaned
                    }
                }
            }
        }
        \KwRet events
    }
    \;
    \Fn{\nextEvents{events: Map[string, Event[]]} $\longrightarrow$ [float, Event[]]}{
        nextEvents: [float, Event[]] $\longleftarrow$ [$\infty$, []]\\
        \For{pair in events}{
            \For{event in pair.second} {
                \If{event.timestamp < nextEvents.first}{
                    nextEvents $\longleftarrow$ [event.timestamp, [event]]
                }
                \ElseIf{event.timestamp == nextEvents.first}{
                    nextEvents $\longleftarrow$ append(event, nextEvents.second)
                }
            }
        }
        \KwRet nextEvent
    }
    \;
    \Fn{\subtractTime{events: Map[string, Event[]], time: float} $\longrightarrow$ Map[string, Event[]]}{
        \For{pair in events}{
            \For{event in pair.second}{
                event.timestamp $\longleftarrow$ event.timestamp - time
            }
        }
        \KwRet events
    }
    \caption{Algorithmus zum Aufbau eines physikalischen Systems - Part 2}
    \label{alg:physikalisches_system_verbesserung_algorithmus_2}
\end{algorithm}