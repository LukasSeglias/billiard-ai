\section{Herleitung Startgeschwindigkeit auf Basis bekannter Endgeschwindigkeit unter Einbezug von Reibung}\label{anhang:herleitung:StartgeschwAnhandEndgeschwMitReibung}
Dies erfordert den Energieerhaltungssatz. Dazu wird die kinetische Energie vor sowie nachher betrachtet.
\begin{align}
    E_{kin} = \frac{1}{2} \cdot m \cdot \norm{\vec{v}}^2\\
    E_{vorher} = \frac{1}{2} \cdot m \cdot \norm{\vec{v_1}}^2\\
    E_{nachher} = \frac{1}{2} \cdot m \cdot \norm{\vec{v_2}}^2
\end{align}
Die Reibung $F_R$ wird mithilfe der Normalkraft $F_N$ und dem Widerstandskoeffizienten $\mu$ definiert\cite{wiki.rollreibung:1}.
Mithilfe der Masse $m$ der Kugel und der Schwerebeschleunigung $g$ kann die Normalkraft $F_N$ berechnet werden.
\begin{align}
    F_R = \mu \cdot F_N\\
    F_N = m \cdot g
\end{align}
Da die Kugel nicht von Beginn an rollt, sondern zuerst gleitet, besteht die Reibung nicht nur aus Roll- $F^r_R$ sondern auch aus
Gleitreibung $F^g_R$.
\begin{align}
    F_R = F^r_R + F^g_R
\end{align}
Es gibt daher auch zwei Widerstandskoeffizienten für die jeweiligen Reibungen. Für die Rollreibung ist dies $\mu_r$ und
für die Gleitreibung $\mu_g$.
Die Reibung $F_R$ wird über eine bestimmte Strecke $\Delta s$, welche die Kugel zurücklegt, angewendet.
Dadurch entsteht eine Arbeit $E_{Reibung}$\cite{wiki.arbeit_physik:1}:
\begin{align}
    E_{Reibung} = F_R \cdot \Delta s
\end{align}
Da die beiden Reibungen nur auf bestimmten Teilstrecken relevant sind, werden diese entsprechend aufgeteilt.
\begin{align}
    \Delta s = \Delta_r s + \Delta_g s\\
    E^r_{Reibung} = F^r_R \cdot \Delta_r s\\
    E^g_{Reibung} = F^g_R \cdot \Delta_g s\\
\end{align}
Ein relevanter Zusatz vor der Herleitung ist die Tatsache, dass die Strecke $\Delta_g s$ des Gleitens von der initialen
Startgeschwindigkeit abhängig ist, welche in diesem Schritt berechnet werden soll. Diese Abhängigkeit wird in Kapitel \ref{anhang:herleitung:gleitdistanz} beschrieben
und hergeleitet. Die wichtigsten Erkenntnisse lauten:
\begin{align}
    \Delta_g s = \frac{12}{49} \cdot \frac{v_0^2}{g \cdot \mu_g}\\
    \Delta_r s = \Delta s - \Delta_g s
\end{align}
Nun kann der Energieerhaltungssatz angewendet werden.
\begin{align}
    E_{vorher} = E_{nachher} + E_{Reibung}\\
    E_{vorher} = E_{nachher} + E^r_{Reibung} + E^g_{Reibung}\\
    \frac{1}{2} \cdot m \cdot \norm{\vec{v_1}}^2 = \frac{1}{2} \cdot m \cdot \norm{\vec{v_2}}^2 + F^r_R \cdot \Delta_r s + F^g_R \cdot \Delta_g s\\
    m \cdot \norm{\vec{v_1}}^2 = m \cdot \norm{\vec{v_2}}^2 + 2 \cdot F^r_R \cdot \Delta_r s + 2 \cdot F^g_R \cdot \Delta_g s\\
    \norm{\vec{v_1}}^2 = \frac{m \cdot \norm{\vec{v_2}}^2 + 2 \cdot F^r_R \cdot \Delta_r s + 2 \cdot F^g_R \cdot \Delta_g s}{m}\\
    \norm{\vec{v_1}}^2 = \frac{m \cdot \norm{\vec{v_2}}^2 + 2 \cdot m \cdot g \cdot \mu_r \cdot \Delta_r s + 2 \cdot m \cdot g \cdot \mu_g \cdot \Delta_g s}{m}\\
    \norm{\vec{v_1}}^2 = \norm{\vec{v_2}}^2 + 2 \cdot g \cdot \mu_r \cdot \Delta_r s + 2 \cdot g \cdot \mu_g \cdot \Delta_g s\\
    \norm{\vec{v_1}}^2 = \norm{\vec{v_2}}^2 + 2 \cdot g \cdot \mu_r \cdot (\Delta s - \Delta_g s) + 2 \cdot g \cdot \mu_g \cdot \Delta_g s\\
    \norm{\vec{v_1}}^2 = \norm{\vec{v_2}}^2 + 2 \cdot g \cdot \mu_r \cdot (\Delta s - (\frac{12}{49} \cdot \frac{\norm{\vec{v_1}}^2}{g \cdot \mu_g})) + 2 \cdot g \cdot \mu_g \cdot (\frac{12}{49} \cdot \frac{\norm{\vec{v_1}}^2}{g \cdot \mu_g})\\
    \norm{\vec{v_1}}^2 = \norm{\vec{v_2}}^2 + 2 \cdot g \cdot \mu_r \cdot \Delta s - \frac{24 \cdot g \cdot \mu_r \cdot \norm{\vec{v_1}}^2}{49 \cdot g \cdot \mu_g} + \frac{24 \cdot g \cdot \mu_g \cdot \norm{\vec{v_1}}^2}{49 \cdot g \cdot \mu_g}\\
    \norm{\vec{v_1}}^2 = \norm{\vec{v_2}}^2 + 2 \cdot g \cdot \mu_r \cdot \Delta s - \frac{24 \cdot \mu_r \cdot \norm{\vec{v_1}}^2}{49 \cdot \mu_g} + \frac{24 \cdot \norm{\vec{v_1}}^2}{49}\\
    \norm{\vec{v_1}}^2 +  \frac{24 \cdot \mu_r \cdot \norm{\vec{v_1}}^2}{49 \cdot \mu_g} - \frac{24 \cdot \norm{\vec{v_1}}^2}{49}= \norm{\vec{v_2}}^2 + 2 \cdot g \cdot \mu_r \cdot \Delta s\\
    \norm{\vec{v_1}}^2 \cdot (1 + \frac{24 \cdot \mu_r}{49 \cdot \mu_g} - \frac{24}{49}) = \norm{\vec{v_2}}^2 + 2 \cdot g \cdot \mu_r \cdot \Delta s\\
    \norm{\vec{v_1}}^2 = \frac{\norm{\vec{v_2}}^2 + 2 \cdot g \cdot \mu_r \cdot \Delta s}{1 + \frac{24 \cdot \mu_r}{49 \cdot \mu_g} - \frac{24}{49}}\\
    \norm{\vec{v_1}}^2 = \frac{\norm{\vec{v_2}}^2 + 2 \cdot g \cdot \mu_r \cdot \Delta s}{\frac{49 \cdot \mu_g}{49 \cdot \mu_g} + \frac{24 \cdot \mu_r}{49 \cdot \mu_g} - \frac{24 \cdot \mu_g}{49 \cdot \mu_g}}\\
    \norm{\vec{v_1}}^2 = \frac{\norm{\vec{v_2}}^2 + 2 \cdot g \cdot \mu_r \cdot \Delta s}{\frac{49 \cdot \mu_g + 24 \cdot \mu_r - 24 \cdot \mu_g}{49 \cdot \mu_g}}\\
    \norm{\vec{v_1}}^2 = \frac{\norm{\vec{v_2}}^2 + 2 \cdot g \cdot \mu_r \cdot \Delta s}{\frac{49 \cdot \mu_g + 24 \cdot (\mu_r - \mu_g)}{49 \cdot \mu_g}}\\
    \norm{\vec{v_1}}^2 = \frac{49 \cdot \mu_g \cdot (\norm{\vec{v_2}}^2 + 2 \cdot g \cdot \mu_r \cdot \Delta s)}{49 \cdot \mu_g + 24 \cdot (\mu_r - \mu_g)}\\
    \norm{\vec{v_1}} = \sqrt{\frac{49 \cdot \mu_g \cdot (\norm{\vec{v_2}}^2 + 2 \cdot g \cdot \mu_r \cdot \Delta s)}{49 \cdot \mu_g + 24 \cdot (\mu_r - \mu_g)}}
\end{align}
Damit ist die Startgeschwindigkeit bestimmt. Nun stellt sich noch die Frage nach der Richtung. Diese zeigt in dieselbe,
wie die Endgeschwindigkeit. Daher kann nun die bekannte Länge mit dem normalisieren Vektor $v_2$ multipliziert
werden:
\begin{align}
    \vec{v_1} = \norm{\vec{v_1}} \cdot \frac{\vec{v_2}}{\norm{\vec{v_2}}}
\end{align}
Daraus folgt die Formel:
\begin{align}
    \vec{v_1} = \sqrt{\frac{49 \cdot \mu_g \cdot (\norm{\vec{v_2}}^2 + 2 \cdot g \cdot \mu_r \cdot \Delta s)}{49 \cdot \mu_g + 24 \cdot (\mu_r - \mu_g)}} \cdot \frac{\vec{v_2}}{\norm{\vec{v_2}}}
\end{align}
