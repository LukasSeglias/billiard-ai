\section{Herleitung Startgeschwindigkeit auf Basis bekannter Endgeschwindigkeit unter Einbezug von Reibung}\label{anhang:herleitung:StartgeschwAnhandEndgeschwMitReibung}
Dies erfordert den Energieerhaltungssatz. Dazu wird die kinetische Energie vor sowie nachher betrachtet.
\begin{align}
    E_{kin} = \frac{1}{2} \cdot m \cdot \norm{\vec{v}}^2\\
    E_{vorher} = \frac{1}{2} \cdot m \cdot \norm{\vec{v_1}}^2\\
    E_{nachher} = \frac{1}{2} \cdot m \cdot \norm{\vec{v_2}}^2
\end{align}
Die Rollreibung $F_R$ wird mithilfe der Normalkraft $F_N$ und dem Rollwiderstandskoeffizienten $c_R$ definiert\cite{wiki.rollreibung:1}.
Mithilfe der Masse $m$ der Kugel und der Schwerebeschleunigung $g$ kann die Normalkraft $F_N$ berechnet werden.
\begin{align}
    F_R = c_R \cdot F_N\\
    F_N = m \cdot g
\end{align}
Die Rollreibung $F_R$ wird über eine bestimmte Strecke $\Delta s$, welche die Kugel zurücklegt, angewendet.
Dadurch entsteht eine Arbeit $E_{Reibung}$\cite{wiki.arbeit_physik:1}:
\begin{align}
    E_{Reibung} = F_R \cdot \Delta s
\end{align}
Nun kann der Energieerhaltungssatz angewendet werden:
\begin{align}
    E_{vorher} = E_{nachher} + E_{Reibung}\\
    \frac{1}{2} \cdot m \cdot \norm{\vec{v_1}}^2 = \frac{1}{2} \cdot m \cdot \norm{\vec{v_2}}^2 + F_R \cdot \Delta s\\
    m \cdot \norm{\vec{v_1}}^2 = m \cdot \norm{\vec{v_2}}^2 + 2 \cdot F_R \cdot \Delta s\\
    \norm{\vec{v_1}}^2 = \frac{m \cdot \norm{\vec{v_2}}^2 + 2 \cdot F_R \cdot \Delta s}{m}\\
    \norm{\vec{v_1}}^2 = \frac{m \cdot \norm{\vec{v_2}}^2 + 2 \cdot m \cdot g \cdot c_R \cdot \Delta s}{m}\\
    \norm{\vec{v_1}}^2 = \norm{\vec{v_2}}^2 + 2 \cdot g \cdot c_R \cdot \Delta s\\
    \norm{\vec{v_1}} = \sqrt{\norm{\vec{v_2}}^2 + 2 \cdot g \cdot c_R \cdot \Delta s}
\end{align}
Damit ist die Startgeschwindigkeit bestimmt. Nun stellt sich noch die Frage nach der Richtung. Diese zeigt in dieselbe,
wie die Endgeschwindigkeit. Daher kann nun die bekannte Länge mit dem normalisieren Vektor $v_2$ multipliziert
werden:
\begin{align}
    \vec{v_1} = \norm{\vec{v_1}} \cdot \frac{\vec{v_2}}{\norm{\vec{v_2}}}
\end{align}
Daraus folgt die Formel:
\begin{align}
    \vec{v_1} = \sqrt{\norm{\vec{v_2}}^2 + 2 \cdot g \cdot c_R \cdot \Delta s} \cdot \frac{\vec{v_2}}{\norm{\vec{v_2}}}
\end{align}
