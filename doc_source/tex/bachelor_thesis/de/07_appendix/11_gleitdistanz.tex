\section{Gleitdistanz}\label{anhang:herleitung:gleitdistanz}
Die zurückgelegte Gleitdistanz ohne Rollen einer Kugel, lässt sich mithilfe der Formel zur Bestimmung des Rollzeitpunkts
des Kapitels \ref{anhang:herleitung:event:rollen} herleiten.
Dazu wird die gleichmässig beschleunigte Bewegungsgleichung verwendet.
\begin{align}
    F &= m \cdot a\\
    s &= \frac{1}{2} \cdot a \cdot t^2 + v \cdot t + s_0
\end{align}
Da es sich bei $a$ um eine entgegengesetzte Kraft handelt, wird diese mit $-1$ multipliziert. Weiterhin gilt nun für die
Kraft $F = m \cdot g \cdot \mu$, wobei es sich bei $\mu$ um den Gleitreibungskoeffizienten handelt. Wird die Formel des Rollzeitpunkts
für $t$ eingesetzt, resultiert die folgende Gleichung.
\begin{align}
    s &= -\frac{1}{2} \cdot \frac{F}{m} \cdot (\frac{v_0}{\frac{7}{2} \cdot g \cdot \mu})^2 + v_0 \cdot \frac{v_0}{\frac{7}{2} \cdot g \cdot \mu}\\
    s &= -\frac{1}{2} \cdot \frac{F}{m} \cdot \frac{v_0^2}{\frac{49}{4} \cdot g^2 \cdot \mu^2} + \frac{v_0^2}{\frac{7}{2} \cdot g \cdot \mu}\\
    s &= -\frac{1}{2} \cdot \frac{F}{m} \cdot \frac{4 \cdot v_0^2}{49 \cdot g^2 \cdot \mu^2} + \frac{2 \cdot v_0^2}{7 \cdot g \cdot \mu}\\
    s &= -\frac{F \cdot 4 \cdot v_0^2}{m \cdot 98 \cdot g^2 \cdot \mu^2} + \frac{2 \cdot v_0^2}{7 \cdot g \cdot \mu}\\
    s &= -\frac{m \cdot g \cdot \mu \cdot 4 \cdot v_0^2}{m \cdot 98 \cdot g^2 \cdot \mu^2} + \frac{2 \cdot v_0^2}{7 \cdot g \cdot \mu}\\
    s &= -\frac{4 \cdot v_0^2}{98 \cdot g \cdot \mu} + \frac{2 \cdot v_0^2}{7 \cdot g \cdot \mu}\\
    s &= -\frac{4 \cdot v_0^2}{98 \cdot g \cdot \mu} + \frac{28 \cdot v_0^2}{98 \cdot g \cdot \mu}\\
    s &= \frac{-4 \cdot v_0^2 + 28 v_0^2}{98 \cdot g \cdot \mu}\\
    s &= \frac{24 \cdot v_0^2}{98 \cdot g \cdot \mu}\\
    s &= \frac{24}{98} \cdot \frac{v_0^2}{g \cdot \mu}\\
    s &= \frac{12}{49} \cdot \frac{v_0^2}{g \cdot \mu}
\end{align}
