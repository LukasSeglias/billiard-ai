\section{Herleitung Event - Totaler Energieverlust}\label{anhang:herleitung:event:outOfEnergy}

Dieses Ereignis tritt ein, sobald eine Kugel ausrollt. Der Zeitpunkt wird über die Geschwindigkeitsgleichung der
gleichmässig beschleunigten Bewegung bestimmt.
\begin{align}
    \vec{v(t)} = \vec{a} \cdot t + \vec{v_0}
\end{align}

Bekannt ist, dass der Betrag des Geschwindigkeitsvektors $0$ sein muss.
\begin{align}
    \norm{\vec{v(t)}} = 0\\
    \vec{0} = \vec{a} \cdot t + \vec{v_0}\label{eq:out_of_energy:00}
\end{align}

Um $t$ anhand von bekannten Werten $\vec{a}$ und $\vec{v_0}$ zu ermitteln, gibt es zwei Möglichkeiten. Bei der Ersten
kann die Gleichung \ref{eq:out_of_energy:00} Komponentenweise aufgestellt werden, wobei auch $t$ in zwei verschiedene
Richtungen betrachtet werden muss.
\begin{align}
    0 = a_x \cdot t_x + v_{0,x}\\
    0 = a_y \cdot t_y + v_{0,y}
\end{align}

Damit ergeben sich zwei unabhängige Gleichungen mit zwei unbekannten $t_x$ und $t_y$, die Zeitpunkte, an denen die Kugel in der
entsprechenden Richtung stillsteht.
Angenommen, eine Kugel rollt nur in X-Richtung, dann ist der Zeitpunkt $t_x$, zu dem die Kugel in X-Richtung stillsteht
und der Zeitpunkt $t_y$, zu dem die Kugel in Y-Richtung stillsteht, unterschiedlich.
Angenommen, eine Kugel rollt mit einer Geschwindigkeit in X- und Y-Richtung. Dann entspricht die Beschleunigung $\vec{a}$
einem konstanten Faktor angewendet auf den Geschwindigkeitsvektor.

\begin{align}
    \vec{a} = -\mu \cdot \vec{v_0}
\end{align}

Die beiden Variablen $t_x$ und $t_y$ sind in diesem Fall äquivalent. Eine Unterscheidung auf die beiden Komponenten wird
also nur getroffen, da sich das Problem auf nur eine Dimension beschränken kann. Es ergeben sich die beiden nachfolgenden
Formeln.
\begin{align}
    t_x = \frac{-v_{0,x}}{a_x}\\
    t_y = \frac{-v_{0,y}}{a_y}
\end{align}

Der Zeitpunkt $t$ zu dem die Kugel vollständig stillsteht, ist so wie folgt gegeben:
\begin{align}
    t = \max{(t_x, t_y)}\\
    t = \max{(\frac{-v_{0,x}}{a_x}, \frac{-v_{0,y}}{a_y})}
\end{align}

Die andere Möglichkeit ist die Betrachtung der Gleichung \ref{eq:out_of_energy:00} über die Beträge der Vektoren
$\vec{a}$ sowie $\vec{v_0}$.

Werden die Beträge genommen, dann kann die Gleichung \ref{eq:out_of_energy:00} als eindimensionales Problem aufgefasst und
dementsprechend gelöst werden. Auch hier ist der Vektor $\vec{a}$ abhängig von der Geschwindgkeit $\vec{v_0}$ über einen
konstanten Faktor.
\begin{align}
    v(t) = \norm{\vec{a}} \cdot t + \norm{\vec{v_0}}
\end{align}

Es folgen einige Umformungen, um die Variable $t$ auf eine Seite der Gleichung zu bringen.
\begin{align}
    0 = \norm{\vec{a}} \cdot t + \norm{\vec{v_0}}\\
    -\norm{\vec{v_0}} = \norm{\vec{a}} \cdot t\\
    -\sqrt{v_{0,x}^2 + v_{0,y}^2} = \sqrt{a_x^2 + a_y^2} \cdot t\\
    v_{0,x}^2 + v_{0,y}^2 = a_x^2 + a_y^2 \cdot t^2\\
    \frac{v_{0,x}^2 + v_{0,y}^2}{a_x^2 + a_y^2} = t^2\\
    \sqrt{\frac{v_{0,x}^2 + v_{0,y}^2}{a_x^2 + a_y^2}} = t
\end{align}

Damit stehen zwei Möglichkeiten zur Verfügung.
\begin{align}
    t = \max{(\frac{-v_{0,x}}{a_x}, \frac{-v_{0,y}}{a_y})}\\
    t = \sqrt{\frac{v_{0,x}^2 + v_{0,y}^2}{a_x^2 + a_y^2}}
\end{align}
Verwendet wird die Gleichung \ref{eq:out_of_energy:01}, da angenommen wird, dass das Nehmen eines Maximalwerts zweier
Divisionen effizienter ist, als die Berechnung einer Wurzel einer Division zweier Additionen.