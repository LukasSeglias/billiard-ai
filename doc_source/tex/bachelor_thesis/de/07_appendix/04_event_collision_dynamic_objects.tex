\section{Herleitung Event - Kollision dynamischer Objekte}\label{anhang:herleitung:event:dynamicObjectCollision}
In diesem Kapitel erfolgt die Herleitung zur Bestimmung des Kollisionszeitpunkts zweier dynamischer Objekte. Beim Billard
sind dies zwei Kugeln, welche in Bewegung sind.

Um die Formel herzuleiten, wird die Ortsgleichung der gleichförmig beschleunigten Bewegung betrachtet:
\begin{align}
    \vec{s(t)} = \frac{1}{2} \cdot \vec{a} \cdot t^2 + \vec{v_0} \cdot t + \vec{s_0}\\
    \begin{pmatrix}s(t)_x\\s(t)_y\end{pmatrix} = \frac{1}{2} \cdot \begin{pmatrix}a_x\\a_y\end{pmatrix} \cdot t^2 + \begin{pmatrix}v_{0, x}\\v_{0, y}\end{pmatrix} \cdot t + \begin{pmatrix}s_{0, x}\\s_{0, y}\end{pmatrix}
\end{align}
Diese ist für beide Kugeln unterschiedlich:
\begin{align}
    \vec{s_1(t)} = \frac{1}{2} \cdot \vec{a_1} \cdot t^2 + \vec{v_{1}} \cdot t + \vec{s_{1}}\\
    \vec{s_2(t)} = \frac{1}{2} \cdot \vec{a_2} \cdot t^2 + \vec{v_{2}} \cdot t + \vec{s_{2}}
\end{align}
Da die Objekte zweidimensional sind, also eine bestimmte Fläche aufweisen, können die Gleichungen nicht einfach gleichgesetzt
werden. Der Radius der Kugel muss mitbeachtet werden, weswegen die Gleichungen einen Punkt suchen, wo die
Distanz zwischen ihnen dem Kugeldurchmesser entspricht. Dies ist der Ort, wo die Kugeln im zweidimensionalen
(auch im dreidimensionalen) Raum kollidieren werden. Dazu wird die Parameterform nach $\vec{d}$ bestimmt.
\begin{align}
    \vec{d} = \vec{s_1(t)} - \vec{s_2(t)}\\
    \vec{d} = \frac{1}{2} \cdot \vec{a_1} \cdot t^2 + \vec{v_{1}} \cdot t + \vec{s_{1}} - (\frac{1}{2} \cdot \vec{a_2} \cdot t^2 + \vec{v_{2}} \cdot t + \vec{s_{2}})\\
    \vec{d} = \frac{1}{2} \cdot \vec{a_1} \cdot t^2 + \vec{v_{1}} \cdot t + \vec{s_{1}} - \frac{1}{2} \cdot \vec{a_2} \cdot t^2 - \vec{v_{2}} \cdot t - \vec{s_{2}}\\
    \vec{d} = \frac{1}{2} \cdot (\vec{a_1} - \vec{a_2}) \cdot t^2 + \vec{v_{1}} \cdot t + \vec{s_{1}} - \vec{v_{2}} \cdot t - \vec{s_{2}}\\
    \vec{d} = \frac{1}{2} \cdot (\vec{a_1} - \vec{a_2}) \cdot t^2 + \vec{v_{1}} \cdot t - \vec{v_{2}} \cdot t + \vec{s_{1}} - \vec{s_{2}}\\
    \vec{d} = \frac{1}{2} \cdot (\vec{a_1} - \vec{a_2}) \cdot t^2 + t \cdot (\vec{v_{1}} - \vec{v_{2}}) + \vec{s_{1}} - \vec{s_{2}}\\
    \vec{d} = t^2 \cdot \frac{1}{2} \cdot \begin{pmatrix}a_{1,x} - a_{2,x}\\a_{1,y} - a_{2,y}\end{pmatrix} + t \cdot \begin{pmatrix}v_{1,x} - v_{2,x}\\v_{1,y} - v_{2,y}\end{pmatrix} + \begin{pmatrix}s_{1,x} - s_{2,x}\\s_{1,y} - s_{2,y}\end{pmatrix}\\
\end{align}
Die Parameterform wird nach Komponenten aufgeteilt:
\begin{align}
    \begin{pmatrix}d_{x}\\d_{y}\end{pmatrix} = t^2 \cdot \frac{1}{2} \cdot \begin{pmatrix}a_{1,x} - a_{2,x}\\a_{1,y} - a_{2,y}\end{pmatrix} + t \cdot \begin{pmatrix}v_{1,x} - v_{2,x}\\v_{1,y} - v_{2,y}\end{pmatrix} + \begin{pmatrix}s_{1,x} - s_{2,x}\\s_{1,y} - s_{2,y}\end{pmatrix}\\
    d_x = t^2 \cdot \frac{1}{2} \cdot (a_{1,x} - a_{2,x}) + t \cdot (v_{1,x} - v_{2,x}) + s_{1,x} - s_{2,x}\\
    d_y = t^2 \cdot \frac{1}{2} \cdot (a_{1,y} - a_{2,y}) + t \cdot (v_{1,y} - v_{2,y}) + s_{1,y} - s_{2,y}
\end{align}
Wobei bekannt ist, dass $\vec{d}$ die Länge $D$ hat (entspricht dem Kugeldurchmesser):
\begin{align}
    \norm{\vec{d}} = \sqrt{(d_x)^2 + (d_y)^2} = D\\
    (d_x)^2 + (d_y)^2 = D^2\label{eq:dynamic_object_collision_event:00}
\end{align}
Die einzelnen Komponenten werden wie folgt substituiert:
\begin{align}
    \Delta a_x = a_{1,x} - a_{2,x}\\
    \Delta a_y = a_{1,y} - a_{2,y}\\
    \Delta v_x = v_{1,x} - v_{2,x}\\
    \Delta v_y = v_{1,y} - v_{2,y}\\
    \Delta s_x = s_{1,x} - s_{2,x}\\
    \Delta s_y = s_{1,y} - s_{2,y}
\end{align}
Daraus folgt:
\begin{align}
    d_x = t^2 \cdot \frac{1}{2} \cdot \Delta a_x + t \cdot \Delta v_x + \Delta s_x\\
    d_y = t^2 \cdot \frac{1}{2} \cdot \Delta a_y + t \cdot \Delta v_y + \Delta s_y\\
    d_x^2 = (t^2 \cdot \frac{1}{2} \cdot \Delta a_x + t \cdot \Delta v_x + \Delta s_x)^2\\
    d_x^2 = t^4 \cdot (\frac{1}{2})^2 \cdot \Delta a_x^2 + 2 \cdot t^3 \cdot \frac{1}{2} \cdot \Delta a_x \cdot \Delta v_x + 2 \cdot t^2 \cdot \frac{1}{2} \cdot \Delta a_x \cdot \Delta s_x  + t^2 \cdot \Delta v_x^2 + 2 \cdot t \cdot \Delta v_x \cdot \Delta s_x + \Delta s_x^2\\
    d_x^2 = (\frac{1}{2})^2 \cdot \Delta a_x^2 \cdot t^4 + \Delta a_x \cdot \Delta v_x t^3 + (\Delta a_x \cdot \Delta s_x + \Delta v_x^2) t^2 + 2 \cdot \Delta v_x \cdot \Delta s_x \cdot t + \Delta s_x^2\label{eq:dynamic_object_collision_event:01}
\end{align}
Aus Gleichung \ref{eq:dynamic_object_collision_event:01} folgt für $d_y^2$:
\begin{align}
    d_y^2 = (\frac{1}{2})^2 \cdot \Delta a_y^2 \cdot t^4 + \Delta a_y \cdot \Delta v_y t^3 + (\Delta a_y \cdot \Delta s_y + \Delta v_y^2) t^2 + 2 \cdot \Delta v_y \cdot \Delta s_y \cdot t + \Delta s_y^2\label{eq:dynamic_object_collision_event:02}
\end{align}
Aus Gleichung \ref{eq:dynamic_object_collision_event:00}, \ref{eq:dynamic_object_collision_event:01} und \ref{eq:dynamic_object_collision_event:02} folgt:
\begin{multline}
    D^2 = (\frac{1}{2})^2 \cdot \Delta a_x^2 \cdot t^4 +\\\
    \Delta a_x \cdot \Delta v_x t^3 +\\\
    (\Delta a_x \cdot \Delta s_x + \Delta v_x^2) t^2 +\\\
    2 \cdot \Delta v_x \cdot \Delta s_x \cdot t +\\\
    \Delta s_x^2 +\\
    (\frac{1}{2})^2 \cdot \Delta a_y^2 \cdot t^4 +\\\
    \Delta a_y \cdot \Delta v_y t^3 +\\\
    (\Delta a_y \cdot \Delta s_y + \Delta v_y^2) t^2 +\\\
    2 \cdot \Delta v_y \cdot \Delta s_y \cdot t +\\\
    \Delta s_y^2\\
\end{multline}
Daraus ergibt sich das folgende Polynom vierten Grades:
\begin{multline}
    D^2 = (\frac{1}{2})^2 \cdot (\Delta a_x^2 + \Delta a_y^2) \cdot t^4 +\\\
    (\Delta a_x \cdot \Delta v_x + \Delta a_y \cdot \Delta v_y) \cdot t^3 +\\\
    (\Delta a_x \cdot \Delta s_x + \Delta v_x^2 + \Delta a_y \cdot \Delta s_y + \Delta v_y^2) t^2 +\\\
    2 \cdot (\Delta v_x \cdot \Delta s_x + \Delta v_y \cdot \Delta s_y) \cdot t +\\\
    \Delta s_x^2 + \Delta s_y^2\label{eq:dynamic_object_collision_event:03}\\\
\end{multline}

Die Koeffizienten können als Skalarprodukte von Vektoren repräsentiert werden. Dies geschieht durch Ausnutzung der
Eigenschaft, dass die quadrierte Norm dem Skalarprodukt des Vektors mit sich selbst entspricht.
\begin{align}
    \vec{a} = \begin{pmatrix}a_x\\a_y\end{pmatrix}\\
    \norm{\vec{a}} = \sqrt{a_x^2 + a_y^2}\\
    \norm{\vec{a}}^2 = a_x^2 + a_y^2\\
    \norm{\vec{a}}^2 = \vec{a} \cdot \vec{a}\\
    a_x^2 + a_y^2 = \vec{a} \cdot \vec{a}\label{eq:dynamic_object_collision_event:04}
 \end{align}

Wird \ref{eq:dynamic_object_collision_event:04} auf die Gleichung \ref{eq:dynamic_object_collision_event:03} angewendet, resultieren die nachfolgenden
Eigenschaften.
\begin{align}
    \Delta a_x^2 + \Delta a_y^2 = \vec{\Delta a} \cdot \vec{\Delta a} \\
    \Delta a_x \cdot \Delta v_x + \Delta a_y \cdot \Delta v_y = \vec{\Delta a} \cdot \vec{\Delta v} \\
    \Delta a_x \cdot \Delta s_x + \Delta v_x^2 + \Delta a_y \cdot \Delta s_y + \Delta v_y^2 = \vec{\Delta a} \cdot \vec{\Delta s} + \vec{\Delta v} \cdot \vec{\Delta v} \\
    2 \cdot (\Delta v_x \cdot \Delta s_x + \Delta v_y \cdot \Delta s_y) = 2 \cdot (\vec{\Delta v} \cdot \vec{\Delta s}) \\
    \Delta s_x^2 + \Delta s_y^2 = \vec{\Delta s} \cdot \vec{\Delta s}
\end{align}

Eingesetzt in \ref{eq:dynamic_object_collision_event:03}:
\begin{align}
    D^2 = (\frac{1}{2})^2 \cdot (\vec{\Delta a} \cdot \vec{\Delta a}) \cdot t^4 +
        (\vec{\Delta a} \cdot \vec{\Delta v}) \cdot t^3 +
        (\vec{\Delta a} \cdot \vec{\Delta s} + \vec{\Delta v} \cdot \vec{\Delta v}) t^2 +
        2 \cdot (\vec{\Delta v} \cdot \vec{\Delta s}) \cdot t +
        \vec{\Delta s} \cdot \vec{\Delta s}
\end{align}