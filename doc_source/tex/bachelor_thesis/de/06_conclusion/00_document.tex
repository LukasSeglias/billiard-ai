\chapter{Fazit}\label{kap:fazit}
Die Zielstellung dieser Arbeit war die Implementation eines Systems, welches den Billardspieler auf geeignete Weise bei der Auswahl und Ausführung von Stössen unterstützt.

Die Situation auf dem Billardtisch wird in Form der Kugelpositionen und -farben detektiert und passend zurückprojiziert.
Die Detektion verfügt über eine hohe Genauigkeit und eine gute Performanz, sodass diese in Echtzeit verwendet werden kann.
Dank der Stabilisierung der detektierten Positionen ist die Anzeige stabil im Ruhezustand und trotzdem reaktionsschnell bei Bewegungen.
Trotz der Problematik von möglichen Detektionsfehlern durch die projizierten Augmentationen funktionieren die Detektion und Klassifikation robust.

Die Suche nach Stössen wurde gut diskretisiert, indem diese von den Löchern ausgehend Stösse findet.
Sie ist fähig, direkte Stösse und Stösse über mehrere Banden oder Kugeln zu finden.
Alle evaluierten Stösse werden nach verschiedenen Kriterien bewertet, welche deren Schwierigkeit ausdrücken.
Die Startgeschwindigkeit jedes Stosses wird rückwärts vom Zielpunkt zurück bis zur weissen Kugel physikalisch korrekt, unter Berücksichtigung der Reibung, berechnet.
Dank der implementierten ereignisbasierten analytischen Physiksimulation können gefundene Stösse simuliert und die daraus
folgende Situation abgeleitet werden, um die Suche über mehrere Stösse zu erlauben und damit die Wahl früherer Stösse zu beeinflussen.
Dabei wurden u.a. die Reihenfolge der einzulochenden Kugeln und die Platzierung der farbigen Kugeln regelkonform umgesetzt.
Die Suche wie auch die Simulation berücksichtigen keinen Sidespin, modellieren jedoch den Topspin.

Mithilfe von Animationen werden die Resultate der Suche und Simulation dem Spieler optimal und direkt auf dem Billardtisch angezeigt.
Dies führt dazu, dass die Richtung des Queues und die Stärke des Schlags an den Animationen abgelesen und daran ausgerichtet werden kann.
Ausserdem sieht der Spieler den Ablauf des Stosses vom Anfang bis zum Schluss und die resultierende Spielsituation.

Es wurde eine erweiterbare Anwendung mit Unity erstellt, die verschiedenste Optionen und Funktionalitäten
für die Anzeige und das Debugging des Gesamtsystems anbietet.

Im Infinity-Modus kann der Spieler mit dem Gesamtsystem spielend interagieren, indem das System automatisch erkennt, wann
neue Vorschläge gemacht werden sollen, ohne den Spielfluss zu unterbrechen.
Es wird erkannt, wenn Kugeln in Bewegung sind, neue Kugeln auf dem Tisch platziert werden und Kugeln vom Tisch verschwinden.
Die Interaktion findet ausschliesslich über das Billardspiel statt, wodurch sich der Spieler darauf konzentrieren kann,
die strategischen Überlegungen des Systems nachzuvollziehen und die Ausführung zu üben.

In dieser Arbeit wurden viele Erkenntnisse und Herleitungen zur Implementation der genannten Komponenten dokumentiert, die für ähnliche Arbeiten nützlich sein können.
Es wurde der Vergleich zu verwanden Arbeiten, die ebenfalls praktischer Natur waren, gezogen und innovative Alleinstellungsmerkmale dieser Arbeit gefunden.

Die Ergebnisse dieser Arbeit sind nicht nur theoretischer Natur, sondern wurden mit einem Aufbau an einem echten Billardtisch
erarbeitet und resultieren in einer \emph{Augmented Reality} Anwendung, welche Hardware und Software vereint.

Die ursprünglichen Ziele der Arbeit wurden übertroffen.
Es wurde ein sehr gut funktionierendes Gesamtsystem entwickelt, welches als sehr solide Grundlage für ein Produkt dienen kann.
