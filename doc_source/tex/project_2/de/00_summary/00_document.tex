\chapter{Zusammenfassung}
In dieser Arbeit wird das Vorgehen des Aufbaus eines Grundsystems zur Erkennung von Billardkugeln sowie deren
Projektion zurück auf den Billardtisch beschrieben. Dieser Ablauf wird in dessen Einzelschritten näher beleuchtet.
Zuerst wird der Stand des Tisches über eine Kamera aufgenommen. Danach erfolgt eine Detektion der einzelnen Kugeln.
Aus dieser Detektion wiederum wird die Position auf dem Tisch bestimmt, welche am Schluss über einen Projektor
auf dem Tisch visualisiert wird. Die genannten Überlegungen können im Kapitel \ref{billardAi} nachgeschlagen werden.

Hierbei spielen Messungenauigkeiten eine grosse Rolle. Daher wird ein Konzept in Kapitel \ref{kap:resultate} vorgestellt, wie der
effektive Fehler des Aufbaus bestimmt wird. Anzumerken sei, dass es mehrere Arten von Fehlern gibt, diese aber auch einzeln
vorgestellt und deren Resultate individuell festgehalten werden. Nebst der Bestimmung der Fehler ist es auch wichtig zu wissen,
welche Auswirkungen diese haben. Dazu wurden einige Grundüberlegungen angestellt, welche sich im Anhang \ref{anhang:fehler}
befinden. Diese sind nicht abschliessend und sollen nur einen Eindruck vermitteln, was bei einem gewissen Fehler im schlimmsten
Fall passieren kann.