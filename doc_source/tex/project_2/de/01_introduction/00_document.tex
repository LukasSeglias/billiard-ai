\chapter{Einführung}
Wie vieles andere ist auch das Erlernen des Billardspiels eine schwierige Sache. Es stellen sich Fragen wie \glqq Welche
Kugel soll man anspielen?\grqq{}, \glqq Wie soll man die Kugel anspielen\grqq{} oder \glqq Wie hält man den Queue richtig?\grqq{}.
Darauffolgend gibt es noch diverse weitere Überlegungen, welche den Profi vom Anfänger unterscheiden. Wie in anderen Spielen
auch, ist hier Weitsicht gefragt. Es geht also nicht nur darum, eine Kugel zu versenken, sondern auch den Spielstand so zu
verändern, dass optimal weitergespielt werden kann. Das Stichwort ist im Billard vorallem die Platzierung der weissen Kugel.

Diese Arbeit hat nun nicht den Anspruch, alle aufgeführten Fragestellungen beantworten zu können. In dieser ersten Iteration
geht es vielmehr darum, die Basis zu legen, auf welcher in der Bachelor-Thesis aufgebaut wird. Um überhaupt daran zu denken,
dem Spieler in irgendeiner Form eine Hilfestellung zu geben, muss zuerst der aktuelle Spielstand möglichst präzise
erkannt und ebenso auf dem Tisch visualisiert werden. Ebendieses Vorgehen wird in den nachfolgenden Kapiteln erläutert.