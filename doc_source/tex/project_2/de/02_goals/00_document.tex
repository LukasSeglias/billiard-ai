\chapter{Ziele}
Ins Billard-Spiel einzusteigen ist nicht ganz einfach. Zu Beginn lässt es sich schlecht abschätzen,
welchen Weg eine Kugel nehmen wird, wenn man sie anschlägt und den optimalen Stoss über mehrere Züge hinaus zu planen,
erst recht. Denn bei fortgeschrittenen Spielen ist es oft wichtig, dass die weisse Kugel optimal für den nächsten Stoss
platziert wird.

Es soll ein System entstehen, das dem Spieler den optimalen Stoss vorschlägt basierend auf Kriterien und
einer festgelegten Tiefe des Spielstands. Dazu soll eine Kamera den Spielstand auf dem Billiartisch erkennen, verarbeiten und
dem Spieler Hilfestellungen mittels eines Projektors anzeigen.

In dieser Arbeit soll das Fundament für die Umsetzung dieser Idee gelegt werden. Diese Basisarbeiten setzen sich aus den folgenden
Teilstücken zusammen:
\begin{description}
    \item[Aufsetzen des Projekts] Dazu gehören nebst Überlegungen zur Architektur und der Implementation davon auch
    das Aufsetzen der Dokumentation, welche in Latex geschrieben wird.
    \item[Aufbau des Systems] Dies beinhaltet die Montage und genaue Ausrichtung der Kamera wie auch des Projektors.
    \item[Kalibrierung der Kamera] Bestimmen der intrinsischen Kameramatrix, um genaue Bildanalyse betreiben zu können.
    \item[Erkennung der Kugelpositionen] Die Kugeln und deren Position sollen anhand eines Farbbilds erkannt werden können.
    \item[Übersetzung in internes Koordinatensystem] Mittels Marker, welche am Tisch angebracht werden, sollen die
    auflösungsabhängigen Pixelkoordinaten in ein standardisiertes internes Koordinatensystem überführt werden.
    \item[Anzeige des aktuellen Spielstands] Die detektierten Positionen aller Kugeln sollen mittels eines Projektors
    auf den Tisch projiziert werden.
    \item[Genauigkeitsanalyse] Die Genauigkeit der Erkennung der Kugelpositionen und der Übersetzung in ein
    internes Koordinatensystem soll untersucht werden.
\end{description}

Weitere zu erarbeitende Lösungen, welche nicht in dieser Arbeit umgesetzt wurden:
\begin{description}
    \item[Klassifikation der Kugeln] Wurden die Kugeln erkannt, sollen sie entsprechend der Farbe klassifiziert werden.
    \item[Theoriearbeiten zur Suche eines Stosses] Um einen optimalen Stoss zu finden, bedarf es zunächst einiger
    theoretischen Grundüberlegungen. Dies beinhaltet z.B. einen Algorithmus, um einen Stoss zu finden sowie auch
    physikalisch korrekt zu beschreiben.
    \item[Suche eines einfachen Stosses] Sobald der theoretische Ansatz erarbeitet wurde, soll eine erste einfache
    Suche implementiert werden. Diese soll nur Stösse berücksichtigten, welche eine Kugel direkt in eine Loch spielen.
    \item[Bewertung von Stössen] Jeder gefundene Stoss muss bewertet werden aufgrund verschiedener zu erarbeitenden Kriterien.
    \item[Anzeige von Stössen] Gefundene Stösse sollen über den Projektor dem Spieler angezeigt werden, um diesen
    zu unterstützen.
    \item[Suche indirekter Stösse] Indirekte Stösse über die Bande oder über andere Kugeln sollen gefunden, bewertet und angezeigt werden.
\end{description}

Es ist weiterhin anzumerken, dass es in erster Linie um Snooker-Billard geht. Dies hat mehrere Gründe. Einerseits soll
in dieser Arbeit nicht die Klassifikation der Kugeln im Zentrum stehen, sondern die Suche nach einem optimalen Stoss.
Es wird angenommen, dass dies mit Snooker-Kugeln einfacher ist als mit Pool-Billard-Kugeln. Andererseits wird das
Projekt zusammen mit einem Unternehmen durchgeführt, welches eventuell auch einen kommerziellen Ansatz verfolgen
will. Da grössere Turniere wie Weltmeisterschaften in Snooker ausgetragen werden, kam schnell der Wunsch auf, das
Hauptaugenmerk darauf zu legen. Nichtsdestotrotz wird die Anwendung so abstrakt gehalten, dass sie mit wenig Aufwand
auf Pool-Billard portiert werden könnte. Dies wird aber vorläufig weder in Projekt-2 noch in der darauffolgenden
Bachelor-Thesis von Relevanz sein.

\subsection{Planung}
Die Planung beinhaltet eine Auflistung der Ziele nach Deadline sowie Bearbeiter.

\begin{table}[ht]
        \rowcolors{1}{\seccolor!10}{\seccolor!10} % Rows with 10% of secondary color
        \begin{tabular}{llll}
            \rowcolor{\seccolor!50}
            Ziel & Datum & Bearbeiter\\\bfhmidline
            Evaluation Beamer \& Kamera & 08.03.2021 & Lukas \& Luca\\\bfhmidline
            Überlegungen zum Aufbau & 11.03.2021 & Lukas \& Luca\\\bfhmidline
            Entscheid Beamer- Kameratyp & 11.03.2021 & Lukas \& Luca\\\bfhmidline
            Theorie der Kamera-Kalibrierung erarbeiten & 18.03.2021 & Lukas\\\bfhmidline
            Theorie der Beamer-Kalibrierung erarbeiten & 25.03.2021 & Lukas \& Luca\\\bfhmidline
            Aufbau des Systems & 01.04.2021 & Lukas \& Luca\\\bfhmidline
            Definitive Kalibrierung & 08.04.2021 & Lukas \& Luca\\\bfhmidline
            Kugel erkennen & 15.04.2021 & Lukas\\\bfhmidline
            Fine-Tuning Kugel erkennen & 29.04.2021 & Lukas\\\bfhmidline
            Marker - Transformation in internes Koordinatensystem & 29.04.2021 & Lukas\\\bfhmidline
            Resultate festhalten Erkennung, Klassifikation \& Transformation & 06.05.2021 & Lukas \& Luca\\\bfhmidline
            Anbindung Unity & 06.05.2021 & Luca\\\bfhmidline
            Dokumentieren, Abschliessen & 03.06.2021 & Lukas \& Luca\\\bfhmidline
        \end{tabular}
    \caption{Ziele}
    \label{tab:targets}
\end{table}