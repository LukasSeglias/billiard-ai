\chapter{Ziele}
Ins Billard-Spiel einzusteigen ist nicht ganz einfach. Zu Beginn kann man schlecht abschätzen, welchen Weg eine Kugel
wählen wird, wenn man sie anschlägt und ebenfalls hat man Mühe, den optimalen Stoss zufinden, erst recht, wenn
man weiter in die Tiefe sehen will. Wenn es also wichtig wird, die Weisse nach einem Stoss für den Nächsten optimal
zu platzieren.

Am Ende wird also ein System entstehen, das einem den optimalen Stoss vorschlägt, basierend auf Kriterien und
einer festgelegten Tiefe.

Die Projekt-2 Arbeit hat das grundlegende Ziel der Vorbereitung auf die Bachelor-Thesis. Daher sei zu Beginn erwähnt,
dass möglichst viel vorgearbeitet werden kann, auch wenn dies an dieser Stelle keine explizite Erwähnung findet.

Im Wesentlichen geht es aber vor allem um die zugrunde liegenden Basisarbeiten. Diese setzen sich aus den folgenden
Teilstücken zusammen:
\begin{description}
    \item[Aufsetzen des Projekts] Dazu gehören nebst Überlegungen zur Architektur und der Implementation davon auch
    das Aufsetzen der Dokumentation, welche in Latex geschrieben wird.
    \item[Aufbau des Systems] Dies beinhaltet die Montage der Kamera wie auch des Projektors.
    \item[Kalibrierung der Kamera] Bestimme die intrinsische Transformationsmatrix, um genaue Bildanalyse betreiben zu
    können.
    \item[Erkennung der Kugelpositionen] Die Kugeln sollen einer Position im Pixelkoordinatensystem zugewiesen werden
    können.
    \item[Klassifikation der Kugeln] Wurden die Kugeln erkannt, sollen sie entsprechend der Farbe klassifiziert werden.
    \item[Übersetzung in internes Koordinatensystem] Mittels Marker, welche am Tisch angebracht werden, sollen die
    auflösungsabhängigen Pixelkoordinaten in ein standardisiertes internes Koordinatensystem überführt werden.
    \item[Theoriearbeiten zur Suche eines Stosses] Um einen optimalen Stoss zu finden, bedarf es zunächst einiger
    theoretischen Grundüberlegungen. Dies beinhaltet z.B. einen Algorithmus, um einen Stoss zu finden sowie auch
    physikalisch korrekt zu beschreiben.
    \item[Suche eines einfachen Stosses] Sobald der theoretische Ansatz erarbeitet wurde, soll eine erste einfache
    Suche implementiert werden. Diese soll nur direkte Stösse berücksichtigen. Indirekte über die Bande wie auch über
    andere Kugeln werden vorerst nicht beachtet. Die Ausgabe soll auch nicht unbedingt über den Projektor erfolgen, eine
    textuelle Präsentation soll hier genügen.
\end{description}

Es ist weiterhin anzumerken, dass es in erster Linie um Snooker-Billard geht. Dies hat mehrere Gründe. Einerseits soll
in dieser Arbeit nicht die Klassifikation der Kugeln im Zentrum stehen, sondern die Suche nach einem optimalen Stoss.
Es wird angenommen, dass dies mit Snooker-Kugeln einfacher geht als mit Pool-Billard-Kugeln. Andererseits wird das
Projekt zusammen mit einem Unternehmen durchgeführt, welches eventuell auch einen kommerziellen Ansatz verfolgen
will. Da grössere Turniere wie Weltmeisterschaften in Snooker ausgetragen werden, kam schnell der Wunsch auf, das
Hauptaugenmerk darauf zu legen. Nichtsdestotrotz wird die Anwendung so abstrakt gehalten, dass sie mit wenig Aufwand
auf Pool-Billard portiert werden könnte. Dies wird aber vorläufig weder in Projekt-2 noch in der darauffolgenden
Bachelor-Thesis von Relevanz sein.

\subsection{Planung}
Die Planung beinhaltet eine Auflistung der Ziele nach Deadline sowie Bearbeiter.

\rowcolors{1}{\seccolor!10}{\seccolor!10} % Rows with 10% of secondary color
\begin{tabular}{llll}
    \rowcolor{\seccolor!50}
    Ziel & Datum & Bearbeiter\\\bfhmidline
    Evaluation Beamer \& Kamera & 08.03.2021 & Lukas \& Luca\\\bfhmidline
    Überlegungen zum Aufbau & 11.03.2021 & Lukas \& Luca\\\bfhmidline
    Entscheid Beamer- Kameratyp & 11.03.2021 & Lukas \& Luca\\\bfhmidline
    Theorie der Kamera-Kalibrierung erarbeiten & 18.03.2021 & Lukas\\\bfhmidline
    Beschreibung Such-Algorithmus & 18.03.2021 & Luca\\\bfhmidline
    Theorie der Beamer-Kalibrierung erarbeiten & 25.03.2021 & Lukas \& Luca\\\bfhmidline
    Aufbau des Systems & 01.04.2021 & Lukas \& Luca\\\bfhmidline
    Definitive Kalibrierung & 08.04.2021 & Lukas \& Luca\\\bfhmidline
    Kugel erkennen \& Klassifikation & 15.04.2021 & Luca\\\bfhmidline
    Fine-Tuning Kugel erkennen \& Klassifikation & 29.04.2021 & Luca\\\bfhmidline
    Marker - Transformation in internes Koordinatensystem & 29.04.2021 & Lukas\\\bfhmidline
    Resultate festhalten Erkennung, Klassifikation \& Transformation & 06.05.2021 & Lukas \& Luca\\\bfhmidline
    Einfache Suche implementieren & 03.06.2021 & Lukas \& Luca\\\bfhmidline
\end{tabular}