\subsection{Kamera}\label{kap:camera}

Als Kamera wurde eine Intel RealSense Depth Camera D435 verwendet \cite{intel:realsense_d435}.
Die intrinsischen Kameraparameter des Farbsensors wurden mithilfe des RealSense SDK 2.43.0 \cite{github:realsense_sdk}
ausgelesen. Beim Farbsensor handelt es sich um einen \emph{OmniVision OV2740} \cite{intel:realsense_d435_datasheet}.
Die Auflösungs des Sensors beträgt 1920x1080 und die Pixelgrösse beträgt \SI{1.4}{\micro\metre} \cite{omnivision:ov2740}.

Nachfolgend ist die allgemeine intrinsische Kameramatrix in \emph{column-major order} aufgeführt.
Siehe \cite{matlab:what_is_camera_calibration} für die Erläuterung der Parameter in \emph{row-major order}.

\begin{align}
    \begin{pmatrix}
    f_x & s   & c_x\\
    0   & f_y & c_y\\
    0   & 0   & 1
    \end{pmatrix}
\end{align}

Die konkreten Werte für die Brennweite $(f_x, f_y)$ [Pixel], das optische Zentrum $(c_x, c_y)$ [Pixel]
und der Schiefekoeffizient $s$ sind nachfolgend eingefügt.

\begin{equation}
\begin{pmatrix}
1375.68884 & 0          & 974.842407\\
0          & 1375.85425 & 539.362732\\
0          & 0          & 1
\end{pmatrix}
\end{equation}

Die Parameter für die Modellierung der radialen und tangentialen Verzerrung $k_1$, $k_2$, $k_3$, $p_1$, $p_2$ sind gemäss
RealSense SDK alle 0 und das verwendete Modell wird \emph{Inverse Brown-Conrady} genannt.

