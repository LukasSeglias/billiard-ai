\chapter{Weitere Arbeiten}
Die vorliegende Arbeit dient als Grundlage für die kommende Bachelor-Thesis. Daher gibt es diverse darauf aufbauende
Tätigkeiten. Diese sind unterschiedlich anspruchsvoll und ebenso von verschiedener Wichtigkeit. Daher werden
diese der Relevanz von definitiver Wichtigkeit bis potenzieller Wichtigkeit geordnet.

\section{Klassifikation der Kugeln}
In einem ersten Schritt muss die Klassifikation der Kugeln ergänzt werden. Hierbei geht es
um die Bestimmung der Art einer Kugel, ist es also eine Weisse, Rote oder Blaue. Als
Umsetzungshinweis sei hier der Hue-Kanal des Bildes gegeben.

\section{Erarbeitung des theoretischen physikalischen Modells}
Als Grundlage der Suche nach dem besten Stoss dient ein vereinfachtes theoretisches physikalisches
Modell. Dessen Parameter und Eigenschaften müssen bestimmt und in einen algorithmischen Kontext
verfasst werden. Hierbei sei angemerkt, dass eine Unmenge an möglichen Stössen betrachtet
werden müssen. Es muss also ein Konzept erarbeitet werden, das parallel jeweils nur
die besten Kandidaten verfolgt.

\section{Bestimmen der Heuristiken}
Zu einer erfolgreichen Suche gehört die Definition einer heuristischen Funktion, welche die
Optimalität eines Stosses wie auch dessen Resultat beschreibt. Dies setzt eine gewisse Kenntniss
über das Billardspiel voraus, wobei eventuell auf Wissen von Fachpersonen zugegriffen werden muss.

\section{Implementation der Suche}
Sobald das physikalische Modell, dessen Berechnungen sowie die Heuristiken stehen, kann die Suche
implementiert werden. Um das Vorgehen zu vereinfachen, wurde die Unity-Applikation bereits
entsprechend vorbereitet. Diese ist in der Lage, auf intuitive Weise verschiedenste Spielstatus zu
faken.

\section{Kalibrierung des Projektors}
Aktuell wird der Tisch in Unity über Konfigurationsparameter skaliert und an den korrekten Platz verschoben. Dies
geschieht also manuell. Der grosse Nachteil dabei ist, dass die Ausrichtung des Projektors relevant ist. Ist dieser leicht
schräg montiert oder ein wenig geneigt, dann resultiert automatisch ein Fehler, der applikationsseitig nicht berücksichtigt
wird. Es wäre denkbar, dass über eine kalibrierte Kamera ein Testmuster des Projektors aufgenommen wird und so dessen
Transformationsmatrix bestimmt wird, um eine Kugel sowie den Tisch am korrekten Ort darzustellen. Dieser Umbau wird in
Betracht gezogen, sollte die Genauigkeit nicht hoch genug sein. Als weiterer Vorteil kann die einfachere Installation
und Inbetriebnahme genannt werden.

\section{Automatische Filtermasken}
Ebenfalls automatisieren könnte man die verwendeten Filtermasken bei der Kugeldetektion. Diese sind aktuell statisch
implementiert und daher beleuchtungsabhängig. Sollte bemerkt werden, dass zu häufig Fehler bei der Detektion passieren,
so könnten die Masken ebenfalls in einem Setup-Schritt bestimmt werden. Anzumerken ist, dass die Projektorkalibrierung
und die automatischen Filtermasken spätestens nach der Bachelor-Thesis angestrebt werden. Wie bereits erwähnt handelt
es sich dabei aber hauptsächlich um Optimierungen, die nur vorgängig gemacht werden, wenn es unbedingt nötig erscheint.

\section{Verbesserung der Genauigkeit}
Sollte im Verlaufe der Entwicklung erkannt werden, dass die Präzision des Systems unzureichend
ist, müsste diese verbessert werden. Die genauen Ansätze müssten evaluiert werden.
Weiterhin muss in diesem Zusammenhang zuerst ein genaueres
externes Messgerät gefunden werden, damit überhaupt eine Verbesserung der Genauigkeit nachgewiesen werden kann.

